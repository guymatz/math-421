\documentclass{report}

\input{.tex/preamble}
\input{.tex/macros}
\input{.tex/letterfonts}

\title{
  \Huge{Math 421---The Theory of Single Variable Calculus}
  \\
  Notes
}
\author{\huge{Guy Matz}}
\date{}
\begin{document}

\section{20230619 - Intro to Math Arguments}%
\begin{itemize}
        \item Stamements & Locgivasl Operations

          Statments are sentinves which are eithertrue of false (not both)

          e.g. The number 6 is even

          \underline{OPERATIONS on statements}

          P & Q are staements.  We can modify and compnine these etatements in
          difffferent ways.

          NOT: $\neg$ - The statement "$\neg P$" is true when P is false

          AND $\wedge$ - The statement "$P \wedge Q $" is true when both P and Q

          OR $\vee$ - The statement "$P \vee Q $" is true when at least one of
          P and Q is true.  False when both P \& Q are false

          IF ... THEN $\Longrightarrow $: The Statement "If P, then Q" is true when
          \begin{itemize}
            \item P is true and Q is true
            \item P is False
          \end{itemize}

          False when P is true and Q is false

          $P \Longrightarrow Q$ is equivalent to $(P \wedge Q) \vee (\neg P)$

          IF AND NLY IF\\
          P iff Q ($P \Leftrightarrow Q$ is true when either
         \begin{itemize}
            \item P and Q are both true
            \item P and Q are both false
          \end{itemize}

          \underline{FOR P $\Longrightarrow Q$ we have the following}

          CONVERSE: The converse of is $Q \Longrightarrow P$
          CONTRAPOSITIVE The constrspoasitive is $\neg Q \Longrightarrow \neg P$

          $P \Longrightarrow Q$ is logically equivalent to its contrapositive $(\neg Q \Longrightarrow \neg P)$


      two statements are logivally equivalent if they have the same truth tables.

      \underline{WRITE OUT TRUTH TABLES}

      Compare the truth tables for if..then, contrapositive and $(P \wedge Q) \vee (\neg P)$

      P    Q   $P \wedge Q)$ $\neg P$      $(P \wedge Q) \vee \neg P$ \\
      T    F \\
      T    T \\
      F    T \\
      F    F \\

      \underline{DEFNIITIONS}
      We say an integer n is even if there exists an integer k such that $n = 2k$

      We say an integer n is odd if there exists an integer k such that $n = 2k+1$

      \thm{ $x^2 + 1$ is even } {
        Suppose x is an integer.  If x is odd, then $x^2 + 1$ is even. 
      }
      \myproof {
        Since xx is odd, $x = 2k+1$ ffor some integer k.  The $x^2 + 1 = (2k+1)^2 + 1) = 4k^2 + 2k + 2) = 2(k^2 + k + 1)$.  Since k is an integer, $sk^2  + 2k =1$ is also 
        an integer.  So, $x^2 + 1 = 2m$, where $m = 2k^2 + 2k + 1$.  
        Hence we are done
      }
      \thm{ x is even } {
        For every integer x, x is even iff $x + 1$ is odd 
      }
      \myproof {
        $\Longleftarrow$ First we want to show $x$ is even $\Longrightarrow x+1$ is odd

        Since x is even, $x = 2k$ for some integer k.  So, $x+1 = 2k+1$.  Then by  defn, $x+1$ is odd

        $\Longrightarrow$ Next we want to show $x+1$ is odd $\Longrightarrow $ x is even.

        If x+1 is odd, $x+1 = sk+1$ ffor some integer k.  This measnt that $x = 2k$,
        By defn, x is even
      }

      REMARK: Proof by cases

\end{itemize}

\end{document}
