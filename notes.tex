\documentclass{report}

\input{.tex/preamble}
\input{.tex/macros}
\input{.tex/letterfonts}

\title{
  \Huge{Math 421---The Theory of Single Variable Calculus}
  \\
  Notes
}
\author{\huge{Guy Matz}}
\date{}
\begin{document}

\part{Prologue}

\chapter{Basic Properties of Numbers}
\section*{Properties}
\begin{enumerate}

\item \textbf{Associativity under Addition}: If $a, b$ and $c$ are any numbers,
  then
$$ a + (b + c) = (a + b) + c$$

\item \textbf{Additive Identity}:  If $a$ is any number, then 
$$ a + 0  = 0 + a = a$$

\item \textbf{Additive Inverse}: For every $a,$ there is a number $-a$ such that
$$a + (-a) = (-a) + a = 0$$

\item \textbf{Commutativity under Addition}: If $a$ and $b$ are any numbers, then
$$a + b = b + a$$

\item \textbf{Associativity under Multiplication}: If $a, b$ and $c$ are any numbers, then
$$a \cdot (b \cdot c) = (a \cdot b ) \cdot c$$

\item \textbf{Multiplicative Identity}: If $a$ is any number, then
$$ a \cdot 1 = 1 \cdot a = a$$

\item \textbf{Multiplicative Inverse}: For every number $a \neq 0,$ there is a number $a^{-1}$ such that
$$ a \cdot a^{-1} = a^{-1} \cdot a  = 1$$

\item \textbf{Commutativity under Multiplication}: If $a$ and $b$ are any numbers, then
$$a \cdot b = b \cdot a$$

\item \textbf{Distributive Property}: If $a, b$ and $c$ are any numbers, then
$$ a \cdot (b + c) = a \cdot b + a \cdot c$$

\item \textbf{Trichotomy Law}: For every number $a,$ one and only one of the following holds ($P$ is positive numbers)
    + $a = 0$
    + $a$ is in the collection $P$
    + $-a$ is in the collection $P$

  \item \textbf{Closure Under Addition}: If $a$ and $b$ are in $P,$ then $a+b$ is in $P$

  \item \textbf{Closure Under multiplication}: If $a$ and $b$ are in $P,$ then $a \cdot b$ is in $P$

\end{enumerate}

\section*{Observations}
$$ |a| = \sqrt{a^2} $$
\begin{equation*}
       |a| =
         \begin{dcases}
            a & a \geq 0 \\
            -a &  a < 0
         \end{dcases}
\end{equation*}

\section*{Theorems}
1. For all numbers $a$ and $b,$ we have
$$|a + b| \leq |a| + |b|$$
  Proof by Cases on:
    + $a \geq 0, b \geq 0$
    + $a \geq 0, b \leq 0$
    + $a \leq 0, b \geq 0$
    + $a \leq 0, b \leq 0$

  Or
  $$(|a + b|)^2 = (a+b)^ + 2sb + b^2
                \leq a^2 + 2|a| \cdot |b| + b^2
                = |a|^2 + 2|a| \cdot|b| + |b|^2
                = (|a| + |b|)^2
  $$

\section{Numbers of Various Sorts}
\subsection*{Definitions}
\begin{itemize}
  \item \textbf{Natural Numbers} ($\NN$)---Counting numbers $\{1 ,2, 3, \cdots\}$
	\item \textbf{Principle of Induction}---$f (x)$ is true for all $x$ provided
		\begin{enumerate}
			\item $f (1)$ is true
			\item Whenever $f (k)$ is true, $f (k+1)$ is true
		\end{enumerate}
	\item \textbf{Principle of Induction} with Sets---If $A$ is any set of Natural Numbers and
		\begin{enumerate}
			\item 1 is in $A$
			\item $k+1$ is in A whenever $k$ is in $A$
		\end{enumerate} 
	 $\dots$  Then $A$ is the set of all natural numbers
	 \item \textbf{Well-Ordering Principle}---If  $A$ is a non-null set, then $A$ has a least member
	 \item $!1=!0=1$
	 \item \textbf{Integers} (${\ZZ}$)---$\{-2, -1, 0, 1 , 2\}$
	 \item \textbf{Rational Numbers} (${\QQ}$)---The quotient $m/n$ of integers (with $n \neq 0$)
     \item \textbf{Irrational Numbers}---Numbers represented by infinite decimals
	 \item \textbf{Real Numbers} (${\RR}$)---Rationals + Irrationals
	 \item \textbf{Even Numbers}---Natural numbers of the form $2k$
	 \item \textbf{Odd Numbers}---Natural numbers of the form $2k + 1$
\end{itemize}
\subsection*{Observations}
\begin{itemize}
	\item The square of an even number is even
	\item The square of an odd number is odd
\end{itemize}

\section{Functions}
\subsection*{Definitions}
\begin{itemize}
	\item \textbf{Even Function}: $f (x) = f (-x)$
	\item \textbf{Odd Function}: $f (x) = -f (-x)$
\end{itemize}

\chapter{Numbers Of Various Sorts}%
  \section{Natural Numbers}%
  
  \subsection{Induction}%
    \begin{itemize}
      \item If $A$ is any set and
        \begin{enumerate}
          \item 1 is in $A$
          \item $k+1$ is in A whenever $k$ is in $A$
        \end{enumerate}
        then $A$ is the set of all natural numbers
      \item \textbf{Complete Induction}: If $A$ is a set of natural
          numbers and
        \begin{enumerate}
          \item 1 is in $A$
          \item $k+1$ is in A if $a, \dots, k$ are in $A$
        \end{enumerate}
        then $A$ is the set of all natural numbers
    \end{itemize}

  \section{Integers}
    \begin{itemize}
      \item Denoted $\mathbb{Z}$
      \item Propert P7 (See Basic Properties above) \textbf{Multiplicative Inverse} fails
    \end{itemize}

  \section{Rational Numbers}%
    \begin{itemize}
      \item Denoted $\mathbb{Q}$
      \item The set of all quotients $m/n$ of integers (with $n \neq 0$)
    \end{itemize}


  \section{Real Numbers}%
    \begin{itemize}
      \item Denoted $\mathbb{R}$
      \item All Rationals, plus irrationals
    \end{itemize}

  \section{Observations}%
    \begin{itemize}
      \item Even number are of the form $2k$
      \item Odd number are of the form $2k + 1$
      \item Even numbers have even squares
      \item Even squares have even square roots
      \item Odd numbers have odd squares
      \item Odd squares have odd square roots
    \end{itemize}


\part{Foundations}%
  \chapter{Functions}%
    \begin{itemize}
      \item A \textbf{Function} is a rule which assigns, to each of certain
        real numbers, some other real number
      \item A \textbf{Function} is a collection  of pairs of numbers with the
        following property

        If $(a,b)$ and $(a,c)$ are both in the collection, then $b=c$; 
        i.e., the collection must not contain two different pairs with the
        same first element.
      \item \textbf{Domain}: The set of numbers to which  applies
      \item \textbf{Domain}: If $f$ is a function, the \textbf{domain} of $f$
        is the set of all $a$ or which there is some $b$ such that $a,b$
        is in $f$.  If $a$ is in the domain of $f,$ it follows from the
        definition of a function that there is, in fact, a \textit{unique}
        number $b$ such that $(a,b)$ is in $f$.  The unique $b$ is denoted
        by $f (a)$
      \item $(f + g) (x) = f (x) + g (x)$
      \item $(f \cdot g) (x) = f (x) \cdot g (x)$
      \item $(\frac{f}{g}) (x) = \frac{f (x)}{g (x)}$
      \item $(c \cdot g) (x) = c \cdot g (x),$ for any constant $c$
      \item To show two functions are equal, show The two functions 
        \begin{enumerate}
          \item have the same domain
          \item have the same value at any number in the domain
        \end{enumerate}
      \item \textbf{Composition}: $(f \circ g) (x) = f (g (x))$

          The domain of $f \circ g$ is $\{x : x$ is in the domain $g$
          and $g (x)$ is in the domain $f\}$
    \end{itemize}

  \chapter{Graphs}%
    \begin{itemize}
      \item The \textbf{graph of $f$ in polar coordinates} is the collection
        of all points $P$ with polar coordinates $ (r, \theta)$ satisfying
        $r = f (\theta)$
    \end{itemize}

  \chapter{Limits}%
    \section{Definitions}%
      \dfn{Limit}{
        The function \textbf{$f$ approaches the limit $l$ near $a$} means

          for every $\epsilon > 0$ there is some $\delta > 0$ such that,
          for all $x,$ if $0 < |x-a| < \delta,$ then $|f (x) -l| < \epsilon$
      }

      \dfn{NEGATION of Limit}{
        The function \textbf{$f$ does NOT approach
          the limit $l$ near $a$} means

          there is some $\epsilon > 0$ such that for every $\delta > 0$,
          there is some $x,$ which satisffies $0 < |x-a| < \delta$,
          but not $|f (x) -l| < \epsilon$
      }

      \thm{} {
          A function cannot approach two different limits
          near $a$.  I.e., if $f$ approaches $l$ near $a,$ and $f$ approaches
          $m$ near $a,$ then $l=m$
      }
      \thm{}{
        If $\lim_{x \to a} f (x) = l$ and
          $\lim_{x \to a} g(x) = m,$ then
          \begin{enumerate}
            \item $\lim_{x \to a} (f+g) (x) = l + m$
            \item $\lim_{x \to a} (f \cdot g) (x) = l \cdot m$
            \item $\lim_{x \to a} \frac{f}{g} (x) = \frac{l}{m}$
          \end{enumerate}
        }

    \section{Example}%

    \thm{A function cannot approach two different limits near $a$}
    {

        If $f$ approaches $l$ near $a$, and $f$ approaches $m$ near $a$, then $l=m$
    }

    \myproof {
        Since $f$ approaches $l$ near $a$, we know that for any $\epsilon 
        > 0$ there is some number $\delta_1>0$ such that, for all $x$,
        \[  \text{if } 0 < |x-a| < \delta_1, \text{then } |f(x) -l| < \epsilon \]
          We also know, since $f$ approaches $m$ near $a, $ that there is some
          $\delta_2 >0$ such that, for all $x, $
          \[ \text{if } 0< |x-a| < \delta_2, \text{then } |f(x) -m| < \epsilon \]
          We have had to use to numbers $\delta_1$ and $\delta_2$, since there
          is no guarantee that the $\delta$ which works in one dimension will
          work in the other.  But in fact, it is now easy to conclude that
          for any $\epsilon > 0$ there is some $\delta > 0$ such that, for all
          $x $,
          \[ \text{if } 0 < |x-a| < \delta, \text{then } |f(x) - l| < \epsilon \text{and } |f(x) -m < \epsilon \]
          We simply choose $\delta = \text{min}(\delta_1, \delta_2)$.

          To complete the proof we just have to pick a particular $\epsilon
          > 0$ for which the two conditions
          \[  |f(x) -l| < \epsilon \text{ and } |f(x) -m| < \epsilon \]
          cannot both hold, if $l \neq m$.  The proper choice is suggested
          by figure 16.  If $l \neq m$ so that $l-m| > 0$, we can choose 
          $|l-m|/2$ as our $\epsilon$.  It follows that there is a
          $\delta > 0$ such that for all $x$,
          \[  \text{if } 0 < |x-a| < \delta, \text{then } |f(x) - l| < \frac{|l-m|}{2} \]
          and
          \[  f(x) -m| < \frac{|l-m|}{2} \]

          This implies that for $0 < |x-a| < \delta$ we have
          \[  |l-m| = |l - f(x) + f(x) - m| \leq |l - f(x)| + |f(x)-m| \]

            $$< \frac{|l-m|}{2}  + \frac{|l-m|}{2} $$

            $= |l-m|$
          a contradiction.
    }

\chapter{Continuous Functions}%
  \dfn{Continuity} {A function $f$ is \textbf{continuous at $a$} if 
      \[ \lim_{x \to a} f(x) = f(a) \]

      I.e., $f$ is continuous at $a$ if a limit exists there.
  }

  \thm{ } {
    If $f$ and $g$ are conintuous at $a$, then
    \begin{enumerate}
      \item $f +  g$ is continuous at $a$
      \item $ f \cdot g$  is continuous at $a$
      \item $ 1 / g$  is continuous at $a$
    \end{enumerate}
  }
  \myproof {
    Since $f$ and $g$ are continuous at $a$,
    \[ \lim_{x \to a} f(x) = f(a) \text{ and } \lim_{x \to a} g(x) = g(a)   \]
    This simplifies to 
    \[ \lim_{x \to a} (f+g)(x) = f(a) + g(a) = (f+g)(a) \]
    which is just the assertion that $f+g$ is continuous at $a$
  }

  \thm{ ~ } {
    If $g$  is continuous at $a$ 
  }

\end{document}
