% This is a template for doing homework assignments in LaTeX, cribbed from M. Frenkel (NYU) and A. Hanhart (UW-Madison)

\documentclass{article} % This command is used to set the type of document you are working on such as an article, book, or presenation

\usepackage[margin=1in]{geometry} % This package allows the editing of the page layout. I've set the margins to be 1 inch. 

\usepackage{amsmath, amsfonts}  % The first package allows the use of a large range of mathematical formula, commands, and symbols.  The second gives some useful mathematical fonts.

\usepackage{graphicx}  % This package allows the importing of images

\usepackage[shortlabels]{enumitem} % This package allows for different types of labels in the enumerate environment 


%This allows us to use the theorem and proof environment 
\usepackage{amsthm}
\theoremstyle{plain}
\newtheorem*{theorem*}{Theorem}
\newtheorem{theorem}{Theorem}
\theoremstyle{definition}
\newtheorem*{definition*}{Definition}
\newtheorem*{lemma*}{Lemma}

%Custom commands.  
\newcommand{\abs}[1]{\left\lvert #1 \right\rvert} %absolute value command

%Custom symbols
\newcommand{\Rb}{\mathbb{R}}




\begin{document}

\begin{center}
\Large{\textbf{Assignment \#6}
            
UW-Madison MATH 421} 
\vspace{5pt}
        
\normalsize{  Guy Matz
        
Due: July 20, 2023   }
\vspace{15pt}
        
\end{center}


\noindent\fbox{\textbf{Exercise \#1:}} Assuming that the function $f(x) = e^x$ is continuous, prove that the equation $e^x = 4-x^7$ has a solution. 

\begin{proof} 
  Let $g(x) = x^7 + e^x -4 = 0$.  At $x =0$,
  \[ g(0)=1 + 0 -4 = -3 \]
  At $x = 2$,
  \[ g(x)=e^2 + 2^7-4 \approx 131.4 \]
  We know that $e^x$ is continuous by supposition, and $4 - x^7$ is
  continuous by previous homework.  We also know - by previous homework -
  that the addition o continuous functions is continuous, so $g(x)$ is
  continuous.


  Hence, by Theorem 7.1, there is some point $c$ for which $g(c)=0$
\end{proof} 

\newpage
\noindent\fbox{\textbf{Exercise \#2:}} Suppose $f$ is continuous on $[a,b]$. If $f(x) \neq 0$ for all $x$ in $[a,b]$, then either $f(x) > 0$ for all $x$ in $[a,b]$ or $f(x) < 0$ for all $x$ in $[a,b]$

\begin{proof}[\textbf{Proof by contradiction}]  Towards a contradiction,
  let's assume that $f(x) \leq 0$ for some $x \in [a,b]$ and $ \geq$ for 
  some $y \in [a,b]$.  We also know that $f \neq 0$, so we have $f(x) < 0$
  and $f(y) < 0$.  But by Theorem 7.1 we know that there is a point
  bewteen $x$ and $y$ where $f = 0$.
\end{proof} 

\newpage
\noindent\fbox{\textbf{Exercise \#3:}} Suppose $f(x) = x^n + a_{n-1} x^{n-1} + \dots + a_1 x + a_0$ is a polynomial. Prove:
\begin{proof}[Proof of (a)] 
  Factoring out $x^n$ from $f(x)$ we have
  \[ f(x) = x^n \left(1 + a_{n-1} x^{-1} + \dots + a_1 x + a_0 x^{-n} \right) \]
  The limit of the product is the product of the limits, so let
    \[ g(x) = x^n \]
  and
    \[ h(x) = (1 + a_{n-1} x^{-1} + \dots + a_1 x + a_0 x^{-n} \]
  Then
    \[ \lim_{x \to \infty} f(x) = \lim_{x \to \infty} g(x) \cdot \lim_{x \to \infty} h(x) \]
    For each off the terms in $h(x)$ - except for 1 - we can see that,
    for example and without loss of generality, for $g_1(x) = a_n{-1}$
    we have a limit as $x \to \infty$, since for any $\epsilon$ we can
    find a $\delta$ - in this case, $\frac{a_{n-1}}{\delta}$ such that
    we can get arbitrarily close to 0.  So $\lim_{x \to \infty} g_1(x) =0$
    which means that $\lim_{x \to \infty} g(x) = 1$, since the constant
    - 1 - is all that remains as $x \to \infty$.

    For $h(x) = x^n$ we see that - by the Archimedian Property - 
    there is no $N$ such that
      \[ \forall M, x > N \implies f(x) > M \]
    So $h(x)$ is unbounded, hence $f(x)$ is unbounded.
    And  $\lim_{x \rightarrow-\infty} f(x) = \infty$.
\end{proof} 

\begin{proof}[Proof of (b)] 

  Using the same argument as above we see that if we restrict $x^n$ to
  be an even function, then for all $|x| > 1$, $f(x)$ is an increasing
  function such that
      \[ \forall M, x > N \implies f(x) > M$ \]
\end{proof} 

\begin{proof}[Proof of (c)] 
  If we restrict $x^n$ to be an odd function, then for all $x < -1$,
  $f(x)$ is an increasing negative function such that
      \[ \forall M, x > N \implies f(x) < M$ \]
  So  $\lim_{x \rightarrow-\infty} f(x) = -\infty$.
\end{proof} 



\newpage
\noindent\fbox{\textbf{Exercise \#4:}} Suppose $f$ is continuous on $\Rb$. If $\lim_{x \rightarrow\infty} f(x) = \infty$ and $\lim_{x \rightarrow -\infty} f(x)=-\infty$, then there exists a number $x$ such that $f(x)=0$. 


\begin{proof} 
  Since $\lim_{x \to \infty} f(x) = \infty$, there exists an $N$ such that
  $x > N \implies f(x)>M$, for all M.  So we pick $N+1$, which we know to 
  be positive.

  For $\lim_{x \to -\infty} f(x) = -\infty$, there exists an $Q$ such that
  $x < P \implies f(x) < Q$, for all Q.  So we pick $Q-1$, which we know
  to be negative, and we have
    \[ f(M+1) > 0 > f(Q-1) \]
  By  Theorem 7.1 we know a point $c$ exists such that $f(c) = 0$

\end{proof} 

\newpage
\noindent\fbox{\textbf{Exercise \#5:}} Prove: If $A,B \subset \Rb$ are both non-empty and bounded above, then 
$$
\sup(A \cup B) = \max\{\sup(A), \sup(B)\}.
$$


\begin{proof} (Hint: first show that $ \max\{\sup(A), \sup(B)\}$ is an upper bound of $A \cup B$, then show it is the least upper bound)

  We first want to show that max\{sup(A), sup(B)\} is an upper bound for
  $A \cup B$.  We know that A and B are bounded above, so there is an 
  $\alpha$ and $\beta$ such that $\alpha \geq a, \forall a \in A$
  and a $\beta \geq b, \forall b \in B$.  So the max\{sup(A), sup(B)\} -
  let's call is $\gamma$ is greater than, or equal to, all $c \in C =
  A \cup B$, 
  making $\gamma$ an upper bound for $C$, since C contains only the
  elements in A or B.

  So $\gamma$ comes from the larger of A or B, in which case
  it is less than all other upper bounds, making it the least upper bound
  for C.
\end{proof} 

\newpage
\noindent\fbox{\textbf{Exercise \#6:}} Prove: If $A,B \subset \Rb$ are both non-empty and bounded above, then 
$$
\sup(A \cap B) \leq \min\{\sup(A), \sup(B)\}.
$$


\begin{proof} (Hint: show that $\min\{\sup(A), \sup(B)\}$ is an upper bound of $A \cap B$) 

  If we pick an arbitrary element $x$ from the set $A \cap B$, then
  $x \leq $ sup(A) and $x \leq sup(B)$ which implies that
  $x \leq $ min\{sup(A), sup(B)\}
\end{proof} 

\newpage
\noindent\fbox{\textbf{Exercise \#7:}} Find an example of subsets $A,B \subset \Rb$ which are both non-empty, both bounded above, and 
\[ \sup(A \cap B) < \min\{\sup(A), \sup(B)\}. \]

\begin{proof} 
  \[ A = \{ 2, 4, 6, 8\} \]
  \[ B = \{ 0, 3, 6, 9\} \]
\end{proof}



\newpage
\noindent\fbox{\textbf{Exercise \#8:}} Prove the following lemma from class:

\begin{lemma*} If $f$ is continuous on $[a,b]$ and $f(a) < 0 < f(b)$, then there exist $\delta_1, \delta_2 > 0$ such that 
	\begin{enumerate}
		\item $f$ is negative on $[a,a+\delta_1)$
		\item $f$ is positive on $(b-\delta_2, b]$.
	\end{enumerate}
\end{lemma*}

\begin{proof} (Hint: look at the proof of Theorem 6-3) 

  Consider the case $f(x)<0$. Since $f$ is continuous at $a$, for every
  $\varepsilon_1>0$ there is a $\delta_1>0$ such that, for all $x$,
  \[ \text { if }|x-a|<\delta_1 \text {, then }|f(x)-f(a)|<\varepsilon_1 \]

  By the Stability Theorem we know that if $f(a) < 0$, then there exists
  $\delta_1 > 0$ such that $f(x) < 0$ for all $x \in (a - \delta_1,
  a + \delta_1)$
  So $f$ is negative on $(a, a + \delta_1)$

  Now we look at the case $f(x)>0$. Since $f$ is continuous at $b$, for
  every $\varepsilon_2>0$ there is a $\delta_2>0$ such that, for all $x$,
  \[ \text { if }|x-a|<\delta_2 \text {, then }|f(x)-f(a)|<\varepsilon_2 \]
  Again, by the Stability Theorem we know that if $f(x) > 0$, then there
  exists $\delta_2 > 0$ such that $f(x) < 0$ for all
  $x \in (b - \delta, b + \delta)$

  So $f$ is positive on $(b - \delta_2, b)$


\end{proof} 

\newpage
\noindent\fbox{\textbf{Exercise \#9:}} Suppose $f$ is a function such that $f(x_1) \leq f(x_2)$ whenever $x_1< x_2$. Prove: 
\begin{enumerate}[(a)]
	\item If $A = \{ f(x) : x < a\}$, then $\lim_{x \rightarrow a^-} f(x) = \sup(A)$.
	\item If $B = \{ f(x) : a < x\}$, then $\lim_{x \rightarrow a^+} f(x) = \inf(B)$.
\end{enumerate}

\begin{proof} (Hint for (a): by a theorem in class for every $\epsilon > 0$ there exists some $\alpha \in A$ such that $\sup(A) -\epsilon < \alpha \leq \sup(A)$)
	
  We want to show that $f(a)$ is the limit for the function at $a-$,
  and that sup(A) = $f(a)$.

  We say $\lim_{x \to a-} f(x) = L$ if $\forall \epsilon > 0, \exists
  \delta > 0 $ such that $a - \delta < x -a < a \implies |f(x) -L | < \epsilon$
  So $f(a)$ is the limit of $A$. This is because $A$ is strictly increasing,  so as we approach a- we can always find a $\delta$ around $x$ that
  satisfies the condition above.
  
  Now to show that $f(a)$ is the least upper bound for A we note that 
  the limit of A does not actually lie within A since a is not in the
  domain, however we can get arbitrarily close to a which is good enough!
  Since there is no element with A satifies the condition for an upper
  bound - i.e.  there is no element which is greater than all others in
  the set - the upper bound must lie outside of the set.  $f(a)$ 
  satisfies the conditions for a least upper bound since it is certainly
  an upper bound, but is also lower than all other upper bounds.


  Ok.  So we can see that $\lim_{x \to a-} f(x) = sup(A) = a$

  Oy.  Now on to part b.  I will make the same terrible argument,
  just swapping a for b, etc.  Sorry to have completely mangled this.


\end{proof} 





\newpage
\noindent\fbox{\textbf{Exercise \#10:}} Give a different proof of Theorem 7-1 which uses the set 
$$B = \{x \in [a,b] : a \leq x \leq b \text{ and } f(x) < 0\}$$
instead of the set $A = \{ x \in [a,b] : f \text{ is negative on } [a,x]\}$. 

\begin{proof} (Hint: as in the proof of Theorem 7-1, you want to show that $a < \sup(B) < b$ and then $f(\sup(B)) = 0$)

%	Define the set $B$, shown in Figure 1, as follows:
%$B=\{x: a \leq x \leq b$, and $f$ is negative on the interval $[a, x]\}$.
%Clearly $B \neq \varnothing$, since $a$ is in $B$; in fact, there is some $\delta>0$ such that $B$ contains all points $x$ satisfying $a \leq x<a+\delta$; this follows from Problem 6-16, since $f$ is continuous on $[a, b]$ and $f(a)<0$. Similarly, $b$ is an upper bound for $B$ and, in fact, there is a $\delta>0$ such that all points $x$ satisfying $b-\delta<x \leq b$ are upper bounds for $B$; this also follows from Problem 6-16, since $f(b)>0$.
%
%From these remarks it follows that $B$ has a least upper bound $\alpha$ and that $a<\alpha<b$. We now wish to show that $f(\alpha)=0$, by eliminating the possibilities $f(\alpha)<0$ and $f(\alpha)>0$.
%
%Suppose first that $f(\alpha)<0$. By Theorem 6-3, there is a $\delta>0$ such that $f(x)<0$ for $\alpha-\delta<x<\alpha+\delta$ (Figure 2). Now there is some number $x_0$ in $B$ which satisfies $\alpha-\delta<x_0<\alpha$ (because otherwise $\alpha$ would not be the least upper bound of $B$ ). This means that $f$ is negative on the whole interval $\left[a, x_0\right]$. But if $x_1$ is a number between $\alpha$ and $\alpha+\delta$, then $f$ is also negative on the wholc interval $\left[x_0, x_1\right]$. Therefore $f$ is negative on the interval $\left[a, x_1\right]$, so $x_1$ is in $B$. But this contradicts the fact that $\alpha$ is an upper bound for $B$; our original assumption that $f(\alpha)<0$ must be false.
%
%Suppose, on the other hand, that $f(\alpha)>0$. Then there is a number $\delta>0$ such that $f(x)>0$ for $\alpha-\delta<x<\alpha+\delta$ (Figure 3). Once again we know that there is an $x_0$ in $B$ satisfying $\alpha-\delta<x_0<\alpha$; but this means that $f$ is negative on $\left[a, x_0\right]$,
%which is impossible, since $f\left(x_0\right)>0$. Thus the assumption $f(\alpha)>0$ also leads to a contradiction, leaving $f(\alpha)=0$ as the only possible alternative.
%\end{proof} 


    
\end{document}
