% This is a template for doing homework assignments in LaTeX, cribbed from M. Frenkel (NYU) and A. Hanhart (UW-Madison)

\documentclass{article} % This command is used to set the type of document you are working on such as an article, book, or presenation

\usepackage[margin=1in]{geometry} % This package allows the editing of the page layout. I've set the margins to be 1 inch. 

\usepackage{amsmath, amsfonts}  % The first package allows the use of a large range of mathematical formula, commands, and symbols.  The second gives some useful mathematical fonts.

\usepackage{graphicx}  % This package allows the importing of images

\usepackage[shortlabels]{enumitem} % This package allows for different types of labels in the enumerate environment 


%This allows us to use the theorem and proof environment 
\usepackage{amsthm}
\theoremstyle{plain}
\newtheorem*{theorem*}{Theorem}
\newtheorem{theorem}{Theorem}
\theoremstyle{definition}
\newtheorem*{definition*}{Definition}
\newtheorem*{lemma*}{Lemma}

%Custom commands.  
\newcommand{\abs}[1]{\left\lvert #1 \right\rvert} %absolute value command

%Custom symbols
\newcommand{\Rb}{\mathbb{R}}




\begin{document}

\begin{center}
\Large{\textbf{Assignment \#6}
            
UW-Madison MATH 421} 
\vspace{5pt}
        
\normalsize{  YOUR NAME HERE
        
Due: July 20, 2023   }
\vspace{15pt}
        
\end{center}


\noindent\fbox{\textbf{Exercise \#1:}} Assuming that the function $f(x) = e^x$ is continuous, prove that the equation $e^x = 4-x^7$ has a solution. 

\begin{proof} 
\end{proof} 






\noindent\fbox{\textbf{Exercise \#2:}} Suppose $f$ is continuous on $[a,b]$. If $f(x) \neq 0$ for all $x$ in $[a,b]$, then either $f(x) > 0$ for all $x$ in $[a,b]$ or $f(x) < 0$ for all $x$ in $[a,b]$

\begin{proof} 

\end{proof} 





\noindent\fbox{\textbf{Exercise \#3:}} Suppose $f(x) = x^n + a_{n-1} x^{n-1} + \dots + a_1 x + a_0$ is a polynomial. Prove:
\begin{enumerate}[(a)]
\item $\lim_{x \rightarrow\infty} f(x) = \infty$.
\item If $n$ is even, then  $\lim_{x \rightarrow-\infty} f(x) = \infty$. 
\item If $n$ is odd, then  $\lim_{x \rightarrow-\infty} f(x) = -\infty$.
\end{enumerate}

\begin{proof}[Proof of (a)] 

\end{proof} 

\begin{proof}[Proof of (b)] 

\end{proof} 

\begin{proof}[Proof of (c)] 

\end{proof} 



\noindent\fbox{\textbf{Exercise \#4:}} Suppose $f$ is continuous on $\Rb$. If $\lim_{x \rightarrow\infty} f(x) = \infty$ and $\lim_{x \rightarrow -\infty} f(x)=-\infty$, then there exists a number $x$ such that $f(x)=0$. 


\begin{proof} 
\end{proof} 

\noindent\fbox{\textbf{Exercise \#5:}} Prove: If $A,B \subset \Rb$ are both non-empty and bounded above, then 
$$
\sup(A \cup B) = \max\{\sup(A), \sup(B)\}.
$$


\begin{proof} (Hint: first show that $ \max\{\sup(A), \sup(B)\}$ is an upper bound of $A \cup B$, then show it is the least upper bound)
\end{proof} 

\noindent\fbox{\textbf{Exercise \#6:}} Prove: If $A,B \subset \Rb$ are both non-empty and bounded above, then 
$$
\sup(A \cap B) \leq \min\{\sup(A), \sup(B)\}.
$$


\begin{proof} (Hint: show that $\min\{\sup(A), \sup(B)\}$ is an upper bound of $A \cap B$) 
\end{proof} 

\noindent\fbox{\textbf{Exercise \#7:}} Find an example of subsets $A,B \subset \Rb$ which are both non-empty, both bounded above, and 
$$
\sup(A \cap B) < \min\{\sup(A), \sup(B)\}.
$$


\begin{proof} 
\end{proof} 



\noindent\fbox{\textbf{Exercise \#8:}} Prove the following lemma from class:

\begin{lemma*} If $f$ is continuous on $[a,b]$ and $f(a) < 0 < f(b)$, then there exist $\delta_1, \delta_2 > 0$ such that 
	\begin{enumerate}
		\item $f$ is negative on $[a,a+\delta_1)$
		\item $f$ is positive on $(b-\delta_2, b]$.
	\end{enumerate}
\end{lemma*}

\begin{proof} (Hint: look at the proof of Theorem 6-3) 
\end{proof} 

\noindent\fbox{\textbf{Exercise \#9:}} Suppose $f$ is a function such that $f(x_1) \leq f(x_2)$ whenever $x_1< x_2$. Prove: 
\begin{enumerate}[(a)]
	\item If $A = \{ f(x) : x < a\}$, then $\lim_{x \rightarrow a^-} f(x) = \sup(A)$.
	\item If $B = \{ f(x) : a < x\}$, then $\lim_{x \rightarrow a^+} f(x) = \inf(B)$.
\end{enumerate}

\begin{proof} (Hint for (a): by a theorem in class for every $\epsilon > 0$ there exists some $\alpha \in A$ such that $\sup(A) -\epsilon < \alpha \leq \sup(A)$)
	
\end{proof} 





\noindent\fbox{\textbf{Exercise \#10:}} Give a different proof of Theorem 7-1 which uses the set 
$$B = \{x \in [a,b] : a \leq x \leq b \text{ and } f(x) < 0\}$$
instead of the set $A = \{ x \in [a,b] : f \text{ is negative on } [a,x]\}$. 

\begin{proof} (Hint: as in the proof of Theorem 7-1, you want to show that $a < \sup(B) < b$ and then $f(\sup(B)) = 0$)
	
\end{proof} 


    
\end{document}
