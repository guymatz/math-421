\documentclass{article} % This command is used to set the type of document you are working on such as an article, book, or presenation

%\usepackage[margin=1in]{geometry} % This package allows the editing of the page layout. I've set the margins to be 1 inch. 

\usepackage{amsmath, amsfonts}  % The first package allows the use of a large range of mathematical formula, commands, and symbols.  The second gives some useful mathematical fonts.

\usepackage{graphicx}  % This package allows the importing of images
\usepackage{marvosym}  % Lightning!

%This allows us to use the theorem and proof environment 
\usepackage{enumitem}
\usepackage{amsthm}
\theoremstyle{plain}
\newtheorem*{theorem*}{Theorem}
\newtheorem{theorem}{Theorem}

\theoremstyle{definition}
\newtheorem*{definition*}{Definition}

\newtheoremstyle{case}{}{}{}{}{}{:}{ }{}
\theoremstyle{case}
\newtheorem{case}{Case}

%Custom commands.  
\newcommand{\abs}[1]{\left\lvert #1 \right\rvert} %absolute value command

%Custom symbols
\newcommand{\Rb}{\mathbb{R}}

\begin{document}

\begin{center}
\Large{\textbf{Assignment \#3}
            
UW-Madison MATH 421} % Name of course here
\vspace{5pt}
        
\normalsize{  Guy Matz% Your name here
        \\ Due: June 30, 2023}
\vspace{15pt}
\end{center}

\section*{Exercises}%
\begin{enumerate}[label={\fbox{\textbf{Exercise \#\arabic* :}}}]
\item Prove the following theorem by induction.  

  \begin{theorem*}[Bernoulli's inequality] Suppose $n$ is a natural
    number and $x$ is a real number. If $x > -1$, then 
    \[ (1+x)^n \geq 1 + nx \]
  \end{theorem*} 

  \begin{proof}
    \underline{Base Case: } With $n=1$,
    \begin{align*}
      (1 + x)^1 &\geq 1 + x \\
      1 + x &\geq 1 + x
    \end{align*}

    \underline{Induction Step: } Assuming $(1+x)^n \geq 1 + nx $, we want
    to show 
    \[ (1+x)^{n+1} \geq 1 + (n+1) \cdot x \]
    This is equal to 
      \[ (1+x) \cdot (1+x)^n \geq 1 + n \cdot x  + x\]
      Dividing by $(1+x)$ we get 
      \[ (1+x)^n \geq \frac{1}{1+x} + \frac{1+n}{1+x} \cdot \frac{x}{1+x} \]
      And since we know that $(1+x)^n \geq 1 + nx $, we can show that
        \[ 1 + n x  \geq \frac{1}{1+x} + \frac{n+1}{1+x} \cdot \frac{x}{1+x} \]
        Multiplying by $(1+x)$ on both sides we get
        \[ 1 + x^2 + (n+1) x  \geq 1 + (n+1) \cdot x \]
        Which reduces to
        \[ x^2 \geq 0 \]
      \end{proof}

\newpage
\item Prove the following theorem by Complete induction.  

 \begin{theorem*} Let $a_1,a_2,\dots$ be the sequence where $a_1=2$, $a_2=4$, $a_3=8$, and 
 $$a_n = a_{n-1}+a_{n-2}+a_{n-3}$$ 
 when $n \geq 4$. Then $a_n \leq 2^n$ for all $n \geq 1$. 
 
 \end{theorem*}

  \begin{proof}
    \underline{Base Case: } 
    \begin{align*}
      2^4 &= 16 \geq a_4 = a_3 + a_2 + a_1 = 8 + 4 + 2 = 14\\
      2^5 &= 32 \geq a_5 = a_4 + a_3 + a_2 = 14 + 8 + 4 = 26 \\
      2^6 &= 64 \geq a_6 = a_5 + a_4 + a_3  = 26 + 14 + 8 = 48
    \end{align*}

    \underline{Induction Step: } Let's assume the hypothesis holds for
    k:
    \begin{align*}
      2^{k-2} &\geq a_{k-2} = a_{k-3} + a_{k-4} + a_{k-5} \\
      2^{k-1} &\geq a_{k-1} = a_{k-2} + a_{k-3} + a_{k-4} \\
      2^{k} &\geq a_{k} = a_{k-1} + a_{k-2} + a_{k-3}
    \end{align*}
    So we need to show that
    \[ 2^{k+1} \geq a_{k} + a_{k-1} + a_{k-2} \]
    Subtracting $2^k$ from both sides we get
    \[ 2^{k+1} - 2^k \geq a_{k} + a_{k-1} + a_{k-2} -2^k \]
    \[ 2^k \geq a_{k} + a_{k-1} + a_{k-2} - 2^k\]
    But we know that $2^k \geq a_{k} + a_{k-1} + a_{k-2} $ so
    \[ 2^k \geq 0 \geq a_{k} + a_{k-1} + a_{k-2} - 2^k\]

  \end{proof}
  

\newpage
\item Prove: if $x$ and $y$ are rational numbers, then $xy$ and $x+y$ are rational numbers. 

P: $x, y \in \mathbb{Q}$\\
Q: $xy, x+y \in \mathbb{Q}$

\begin{proof}

  If $x, y \in \mathbb{Q} $ then there exists $\{ \{a,b,j,k\} \in \mathbb{Z}: b,k \neq 0 \}$
such that 
  \[x = \frac{a}{b}, y = \frac{j}{k} \]

  Then $xy = \frac{a}{b} \cdot \frac{j}{k} = \frac{aj}{bk} $ is rational
  since the multiplication of two integers is an integer.

  And $x+y = \frac{a}{b} + \frac{j}{k} = \frac{ak + jb}{bk}$ is
  rational since $bk \neq 0$, and the addition and multiplication 
  of integers is an integer.
\end{proof} 

\newpage
\item Prove: if $x$ is irrational, $y$ is rational, and $y \neq 0$, then $xy$ is irrational. 


P: $x \in \mathbb{I} \wedge y \in \mathbb{Q} \wedge y \neq 0$ \\
$\neg P: x \notin \mathbb{I} \vee y \notin \mathbb{Q} \vee y = 0$ \\
Q: $xy \in \mathbb{I}$ \\
$\neg Q: xy \notin \mathbb{I}$

\renewcommand\qedsymbol{\Lightning}
\begin{proof}[\textbf{Proof by contradiction}]  Towards a contradiction,
  let's assume that $x \notin \mathbb{I}$.  Then we want to show 
  $\neg Q \wedge P \implies \neg P$.  I.e.,
  $xy \notin \mathbb{I} \wedge x \in \mathbb{I} $ leads to a
  contradiction.

  Since $y \in \mathbb{Q}$ then there exists $\{ p,q \in \mathbb{Z}: \}$
  such that
    \[y = \frac{p}{q} \]
  And, Since $xy \in \mathbb{Q}$, there exists $\{ m,n \in \mathbb{Z}: \}$
  such that
    \[ xy = x \cdot \frac{p}{q} = \frac{m}{n} \]
  And so,
  \[ x = \frac{mq}{pn} \] 
  But then $\frac{mq}{pn}$ is rational since the multiplication of two
  integers is an integer.  So $x$ must be irrational.

\end{proof} 
\renewcommand\qedsymbol{$\square$}

\newpage
\item For this problem we need the following definition: 
  \begin{definition*}[ Divisibility ]
    An integer $n$ is \emph{divisible} by an integer $k$ if the ratio $n/k$ is an integer. 
  \end{definition*}
  For example: -3, 0, 3, 6 are all divisible by 3 while 1, 2, 4, 5 are not divisible by 3.  Prove the following: 

\begin{theorem*} Suppose $n$ is an integer. If $n^2$ is divisible by $3$, then $n$ is divisible by $3$. 

\end{theorem*}
\emph{(Hint: if $n$ is not divisible by $3$, then $n=3k+1$ or $n=3k+2$ for some integer $k$.)}

P: $ 3 | n^2 $\\
Q:  $ 3 | n $

\begin{proof}[\textbf{Proof By Contrapositive}]
  Assuming that $3 \not| n$ we want to show that $3 \not| n^2$.

  So, for some integer $k$, we know what either $n = 3k + 1$ or $n=3k + 2$.
    \begin{case}
      $n = 3k + 1$ 

      If $n=3k+1$, then
%      \[ $n^2 = 9k^2 + 6k + 1 = 3(3k^2 + 2k) + 1\]
      This , by our definition, is not divisble by 3

    \end{case}
    \begin{case}
      $n = 3k + 2$

      If $n=3k+2$, then
      \[ n^2 = 9k^2 + 12k + 4 = 3(3k^2 + 4k + 1) + 1 \]
      But this too is not divisble by 3.
    \end{case}

\end{proof}

\renewcommand\qedsymbol{\Lightning}
\begin{proof}[\textbf{Proof by contradiction}]  Towards a contradiction,
  let's assume that \[ 3 \not| n^2 \]

  We also know that $3 | n$, so there is some integer, $k$ such that
  $n = 3k$. Squaring both sides by 
  we get $n^2 = 3^2k^2 = 3(3k^3)$, which is surely divisble by 3.  But
  we said that  $ 3 \not| n^2 $!

\end{proof} 
\renewcommand\qedsymbol{$\square$}

\newpage
\item Prove that $\sqrt{2}+\sqrt{3}$ and $\sqrt{2}-\sqrt{3}$ are both
  irrational numbers.

P: 
Q:

\renewcommand\qedsymbol{\Lightning}
\begin{proof}[\textbf{Proof by contradiction}]  Towards a contradiction,
  let's assume that either, but not both, $\{\sqrt{2}+\sqrt{3},
  \sqrt{2}-\sqrt{3} \} \in \mathbb{Q}$.  Let's say $x = 
  \sqrt{2}+\sqrt{3} \in \mathbb{Q}, y = \sqrt{2}+\sqrt{3} \in \mathbb{I}$.
  Then there exists $m,n \in \mathbb{Z} $ such that $\frac{m}{n} = x
  = \sqrt{2} + \sqrt{3}$.
  Then 
  \[  xy = \frac{m}{n} \cdot y \]
  And
  \[  (\sqrt{2}+\sqrt{3}) \cdot (\sqrt{2}-\sqrt{3})  = -1 \]
  Hence,
  \[  \frac{m}{n} \cdot y  = -1 \]
  and
  \[  \cdot y  = -\frac{n}{m}  \]
  But we said that $y$ is irrational!!

\end{proof} 
\renewcommand\qedsymbol{$\square$}
\newpage
\item Prove that if $x$ satisfies 
  \[ x^n+a_{n-1}x^{n-1}+...+a_{0}=0, \]
  for some integers $a_{n-1},...,a_0$, then $x$ is irrational unless $x$ is an integer (i.e $x$ is rational if and only if it is an integer).

P: x satisfies $x^n+a_{n-1}x^{n-1}+...+a_{0}=0$ ,for some integers $a_{n-1},...,a_0$
Q: $x \in \mathbb{I} \vee x \in \mathbb{Z} $

  \renewcommand\qedsymbol{\Lightning}
  \begin{proof}[\textbf{Proof by contradiction}]  Towards a contradiction,
    Let's assume that there is some rational number, $x = p/q$, where
    gcd($p,q) = 1$, which is a solution to this equation.

    The equation becomes
    \[
      \left( \frac{p}{q}\right) ^n
        + a_{n-1}\left( \frac{p}{q}\right) ^{n-1}
        + \dots + a_{1}\left( \frac{p}{q}\right)^{1}
        + a_0 = 0
      \]
      Subtracting $a_0$, we get 
    \[
      \left( \frac{p}{q}\right) ^n
        + a_{n-1}\left( \frac{p}{q}\right) ^{n-1}
        + \dots + a_{1}\left( \frac{p}{q}\right)^{1}
        = - a_0
      \]
      Factoring out $\left( \frac{p}{q} \right)^n$ we have
    \[
      \left( \frac{p}{q}\right) ^n \left(
        1 + a_{n-1}\left( \frac{p}{q}\right) ^{-1}
        + \dots +
        a_{1}\left( \frac{p}{q}\right)^{1-n}
        \right)
        = - a_0
      \]
      And now we multiply both sides by $q^n$ and mu;ltiply by $p^n$
      on the lefft to getto get 
    \[
        p^n + a_{n-1}p^{n-1}q
        + \dots +
        a_{1} p q^{n-1}
        + a_0 q^n  = 0
      \]
      And factoring out $p$ we get
    \[
        p^n + q ( a_{n-1}p^{n-1}
        + \dots +
        a_{1} p q^{n-2}
        + a_0 q^{n-1} )  = 0
      \]
      Alright!  Now we're getting close to our contradiction!  Now
      we say
      \[ m = 
        a_{n-1}p^{n-1}
        + \dots +
        a_{1} p q^{n-2}
        + a_0 q^{n-1}
      \]
      $m$ is an integer because it's the product of integer.
      So we have $p^n = -qm$, or $\frac{p^2}{q} = -m$.
      OK, OK!  So remember how we said that gcd($p,q)=1$?  Well
      then how can $\frac{p^2}{q}$ be equal to an integer?!  It can't!
      BAM!

  \end{proof} 
  \renewcommand\qedsymbol{$\square$}
\newpage
\item (Optional) Prove the binomial theorem using induction. This states that for all $x,y \in \mathbb{R}$ and $n\geq 1$,
  \[ (x+y)^n = \sum_{r=0}^{n} \binom{n}{r} x^{n-r} y^{r}, \]
      where $\binom{n}{r} = \frac{n!}{(n-r)! r!}$. 

    \emph{Hint: Assume the fact }
    \[ \binom{n}{r}+ \binom{n}{r-1} = \binom{n+1}{r} \]
    If possible, can you give a reasoning for this equality?)

P: 
Q:

\begin{proof}
\end{proof} 

\newpage
\end{enumerate}

\newpage

\section*{Definitions}

\section*{Theorems}

\section*{Properties}
  \begin{itemize}
        \item
  \end{itemize}

\end{document}
