\documentclass{article} % This command is used to set the type of document you are working on such as an article, book, or presenation

%\usepackage[margin=1in]{geometry} % This package allows the editing of the page layout. I've set the margins to be 1 inch. 

\usepackage{amsmath, amsfonts}  % The first package allows the use of a large range of mathematical formula, commands, and symbols.  The second gives some useful mathematical fonts.

\usepackage{graphicx}  % This package allows the importing of images
\usepackage{marvosym}  % Lightning!

%This allows us to use the theorem and proof environment 
\usepackage{enumitem}
\usepackage{amsthm}
\theoremstyle{plain}
\newtheorem*{theorem*}{Theorem}
\newtheorem{theorem}{Theorem}

\theoremstyle{definition}
\newtheorem*{definition*}{Definition}

\newtheoremstyle{case}{}{}{}{}{}{:}{ }{}
\theoremstyle{case}
\newtheorem{case}{Case}

%Custom commands.  
\newcommand{\abs}[1]{\left\lvert #1 \right\rvert} %absolute value command

%Custom symbols
\newcommand{\Rb}{\mathbb{R}}

\begin{document}

\begin{center}
\Large{\textbf{Assignment \#3}
            
UW-Madison MATH 421} % Name of course here
\vspace{5pt}
        
\normalsize{  Guy Matz% Your name here
        \\ Due: June 29, 2023}
\vspace{15pt}
\end{center}

\section*{Exercises}%
\begin{enumerate}[label={\fbox{\textbf{Exercise \#\arabic* :}}}]
\item Prove the following theorem by induction.  

\begin{theorem*}[Bernoulli's inequality] Suppose $n$ is a natural
  number and $x$ is a real number. If $x > -1$, then 
  \[ (1+x)^n \geq 1 + nx \]

\end{theorem*} 

  \proof {~\\
    \underline{Base Case: } With $n=1$,
    \begin{align*}
      (1 + x)^1 &\geq 1 + x \\
      1 + x &\geq 1 + x
    \end{align*}

    \underline{Induction Step: } Assuming $(1+x)^n \geq 1 + nx $, we want
    to show 
    \[ (1+x)^{n+1} \geq 1 + (n+1) \cdot x \]
    This is equal to 
      \[ (1+x) \cdot (1+x)^n \geq 1 + n \cdot x  + x\]
      Dividing by $(1+x)$ we get 
      \[ (1+x)^n \geq \frac{1}{1+x} + \frac{n+1}{1+x} \cdot \frac{x}{1+x} \]
      And now we can show that
        \[ 1 + n x  \geq \frac{1}{1+x} + \frac{n+1}{1+x} \cdot \frac{x}{1+x} \]
        Multiplying by $(1+x)$ on both sides we get
        \[ 1 + x^2 + (n+1) x  \geq 1 + (n+1) \cdot x \]
        Which reduces to
        \[ x^2 \geq 0 \]
  }

\newpage
\item Prove the following theorem by Complete induction.  

 \begin{theorem*} Let $a_1,a_2,\dots$ be the sequence where $a_1=2$, $a_2=4$, $a_3=8$, and 
 $$a_n = a_{n-1}+a_{n-2}+a_{n-3}$$ 
 when $n \geq 4$. Then $a_n \leq 2^n$ for all $n \geq 1$. 
 
 \end{theorem*}

P: 
Q:

\begin{proof}
\end{proof} 

\newpage
\item Prove: if $x$ and $y$ are rational numbers, then $xy$ and $x+y$ are rational numbers. 

P: $x, y \in \mathbb{Q}$\\
Q: $xy, x+y \in \mathbb{Q}$

\begin{proof}

  If $x, y \in \mathbb{Q} $ then there exists $\{ \{a,b,j,k\} \in \mathbb{Z}: b,k \neq 0 \}$
such that 
  \[x = \frac{a}{b}, y = \frac{j}{k} \]

  Then $xy = \frac{a}{b} \cdot \frac{j}{k} = \frac{aj}{bk} $ is rational
  since the multiplication of two integers is an integer.

  And $x+y = \frac{a}{b} + \frac{j}{k} = \frac{ak + jb}{bk} $, which is
  rational, since $bk \neq 0$ and the addition and multiplication 
  of integers is an integer.
\end{proof} 

\newpage
\item Prove: if $x$ is irrational, $y$ is rational, and $y \neq 0$, then $xy$ is irrational. 


P: $x \in \mathbb{I} \wedge y \in \mathbb{Q} \wedge y \neq 0$ \\
Q: $xy \in \mathbb{I}$

\renewcommand\qedsymbol{\Lightning}
\begin{proof}[\textbf{Proof by contradiction}]  Towards a contradiction,
  let's assume that $x \notin \mathbb{I}$.  Then we want to show that
  $x \notin \mathbb{I} \wedge xy \in \mathbb{I}$ leads to a
  contradiction.

  If both $x,y \in \mathbb{Q} $ then there exists $\{ \{a,b,j,k\} \in \mathbb{Z}: b,k \neq 0 \}$ such that
    \[x = \frac{a}{b}, y = \frac{j}{k} \]
  Then $xy = \frac{a}{b} \cdot \frac{j}{k} = \frac{aj}{bk} $.  But this
  is rational since the multiplication of two integers is an integer.
\end{proof} 
\renewcommand\qedsymbol{$\square$}

\newpage
\item For this problem we need the following definition: 
  \begin{definition*}[ Divisibility ]
    An integer $n$ is \emph{divisible} by an integer $k$ if the ratio $n/k$ is an integer. 
  \end{definition*}
  For example: -3, 0, 3, 6 are all divisible by 3 while 1, 2, 4, 5 are not divisible by 3.  Prove the following: 

\begin{theorem*} Suppose $n$ is an integer. If $n^2$ is divisible by $3$, then $n$ is divisible by $3$. 

\end{theorem*}
\emph{(Hint: if $n$ is not divisible by $3$, then $n=3k+1$ or $n=3k+2$ for some integer $k$.)}

P: 
Q:

\begin{proof}
\end{proof} 

\newpage
\item Prove that $\sqrt{2}+\sqrt{3}$ and $\sqrt{2}-\sqrt{3}$ are both  irrational numbers.

P: 
Q:

\begin{proof}
\end{proof} 

\newpage
\item Prove that if $x$ satisfies 
  \[ x^n+a_{n-1}x^{n-1}+...+a_{0}=0, \]
  for some integers $a_{n-1},...,a_0$, then $x$ is irrational unless $x$ is an integer (i.e $x$ is rational if and only if it is an integer).

P: 
Q:

\begin{proof}
\end{proof} 

\newpage
\item (Optional) Prove the binomial theorem using induction. This states that for all $x,y \in \mathbb{R}$ and $n\geq 1$,
%  \[ (x+y)^n = \sum_{r=0}^{n} \binom{n}{r} x^{n-r} y^{r}, \]
%      where $\binom{n}{r} = \frac{n!}{(n-r)! r!}$. 
%      \emph{Hint: Assume the fact $\binom{n}{r}+ \binom{n}{r-1} = \binom{n+1}{r}$. If possible, can you give a reasoning for this equality?)}


P: 
Q:

\begin{proof}
\end{proof} 

\newpage
\end{enumerate}

\newpage

\section*{Definitions}

\section*{Theorems}

\section*{Properties}
  \begin{itemize}
        \item
  \end{itemize}

\end{document}
