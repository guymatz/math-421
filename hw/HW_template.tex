% This is a template for doing homework assignments in LaTeX, cribbed from M. Frenkel (NYU) and A. Hanhart (UW-Madison)

\documentclass{article} % This command is used to set the type of document you are working on such as an article, book, or presenation

\usepackage[margin=1in]{geometry} % This package allows the editing of the page layout. I've set the margins to be 1inch. 

\usepackage{amsmath, amsfonts}  % The first package allows the use of a large range of mathematical formula, commands, and symbols.  The second gives some useful mathematical fonts.





%Custom commands 
\newcommand{\der}[2]{\frac{\mathrm{d} #1}{\mathrm{d} #2}}

%Custom symbols
\newcommand{\Rb}{\mathbb{R}}







\begin{document}

\begin{center}
\Large{\textbf{Assignment \#1}
            
UW-Madison MATH 421} % Name of course here
\vspace{5pt}
        
\normalsize{  YOUR NAME HERE% Your name here
        
Due:  June 20, 2023 at 11:59pm}        
\vspace{15pt}
        
\end{center}


    
\subsection*{Exercise One:} 
    
a) Code:
\begin{verbatim}
If $f(x) = x^n$,  then 
$$
f^\prime(x) = n x^{n-1}.
$$
\end{verbatim}

Output: If $f(x) = x^n$,  then 
$$
f^\prime(x) = n x^{n-1}.
$$

 b)

 c)

    
\subsection*{Exercises Two:}     
    
\subsection*{Exercise Three:}

\subsection*{Exercise Four:}

\subsection*{Exercise Five:}

\subsection*{Exercise Six:}


    
    
\end{document}
