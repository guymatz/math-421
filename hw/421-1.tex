% This is a template for doing homework assignments in LaTeX, cribbed from M. Frenkel (NYU) and A. Hanhart (UW-Madison)

\documentclass{article} % This command is used to set the type of document you are working on such as an article, book, or presenation

\usepackage[margin=1in]{geometry} % This package allows the editing of the page layout. I've set the margins to be 1inch. 

\usepackage{amsmath, amsfonts}  % The first package allows the use of a large range of mathematical formula, commands, and symbols.  The second gives some useful mathematical fonts.
\usepackage[shortlabels]{enumitem}  % The first package allows the use of a large range of mathematical formula, commands, and symbols.  The second gives some useful mathematical fonts.

%Custom commands 
\newcommand{\der}[2]{\frac{\mathrm{d} #1}{\mathrm{d} #2}}

%Custom symbols
\newcommand{\Rb}{\mathbb{R}}

\begin{document}

\begin{center}
\Large{\textbf{Assignment \#1}
            
UW-Madison MATH 421} % Name of course here
\vspace{5pt}

\normalsize{  Guy Matz% Your name here

Due:  June 20, 2023 at 11:59pm}        
\vspace{15pt}

\end{center}


\subsection*{Exercise One:} 
    
\begin{enumerate}[(a)]
  \item Code:
\begin{verbatim}
If $f(x) = x^n$,  then 
$$
f^\prime(x) = n x^{n-1}.
$$
\end{verbatim}

Output:

If $f(x) = x^n$,  then 
$$
f^\prime(x) = n x^{n-1}.
$$
\item Code:
\begin{verbatim}
If $n \neq -1$, then
   \[ \int_{{}}^{{}} {x^n} \: d{x} {}  = \frac{1}{n+1} x^{n+1} + C \]
\end{verbatim}

Output:

If $n \neq -1$, then
   \[ \int_{{}}^{{}} {x^n} \: d{x} {}  = \frac{1}{n+1} x^{n+1} + C \]

\item Code:
\begin{verbatim}
  The derivative of the function $f$ at $x=a$ is
  \[ f'(a) = \lim_{h \to 0 } \frac{f(a + h) - f(a)}{h}  \]
\end{verbatim}
Output:

  The derivative of the function $f$ at $x=a$ is
  \[ f'(a) = \lim_{h \to 0 } \frac{f(a + h) - f(a)}{h}  \]

\end{enumerate}

\newpage

\subsection*{Exercise Two:}
Using \textbackslash{}right and \textbackslash{}left commands, recreate the followings statements (remember to include the code!):

\begin{enumerate}[(a)]
  \item The number $e$ is defined by

    Code:
    \begin{verbatim}
      The number $e$ is defined by
      \[ e = \lim_{n \to \infty} \left( 1 + \frac{1}{n} \right)^n \]
    \end{verbatim}
    Output:

      The number $e$ is defined by
      \[ e = \lim_{n \to \infty} \left( 1 + \frac{1}{n} \right)^n \]
  \item If $f$ is a continuous function

    Code:
    \begin{verbatim}
      If $f$ is a continuous function, then
      \[ \frac{d}{dx} \left[ \int_{{a}}^{{x}} {f(t)} \: d{t} \right] = f(x) \]
    \end{verbatim}
    Output:

      If $f$ is a continuous function, then
      \[ \frac{d}{dx} \left[ \int_{{a}}^{{x}} {f(t)} \: d{t} \right] = f(x) \]
\end{enumerate}


\newpage
\subsection*{Exercise Three:}

Code:
\begin{verbatim}
  \[ 
    \det \begin{pmatrix}
      a & b \\
      c & d
    \end{pmatrix} = ad - bc
  \]
\end{verbatim}

Output:
  \[ 
    \det
    \begin{pmatrix}
      a & b \\
      c & d
    \end{pmatrix} = ad - bc
  \]


\newpage
\subsection*{Exercise Four:}

Code:
\begin{verbatim}
  If $f(x) = x^2$, then
  \begin{align*}
    f'(a) &= \lim_{h \to 0} \frac{f(a+h) - f(h)}{h} \\
          &= \lim_{h \to 0} \frac{(a + h)^2 - s^2}{h} \\
          &= \lim_{h \to 0} \frac{a^2 + 2ah + a^2 - a^2}{h} \\
          &= \lim_{h \to 0} (2a + h) = 2a
  \end{align*}
\end{verbatim}

Output:

  If $f(x) = x^2$, then
  \begin{align*}
    f'(a) &= \lim_{h \to 0} \frac{f(a+h) - f(h)}{h} \\
          &= \lim_{h \to 0} \frac{(a + h)^2 - s^2}{h} \\
          &= \lim_{h \to 0} \frac{a^2 + 2ah + a^2 - a^2}{h} \\
          &= \lim_{h \to 0} (2a + h) = 2a
  \end{align*}

\newpage
\subsection*{Exercise Five:}
Code:
\begin{verbatim}
  If
  \[ 
  f(x) = \begin{cases}
    x      & \text{ if $ x \leq 1$ } \\
    2x + 1 & \text{ if $ x > 1 $ }
  \end{cases}
  \]
  then $\lim_{x \to 1} f(x)$ does not exist
\end{verbatim}

Output:

  If
  \[ 
  f(x) = \begin{cases}
    x      & \text{ if $ x \leq 1$ } \\
    2x + 1 & \text{ if $ x > 1 $ }
  \end{cases}
  \]
  then $\lim_{x \to 1} $ does not exist

\newpage
\subsection*{Exercise Six:}

Code:
\begin{verbatim}
  \newcommand{\pder}[2]{\frac{\partial #1}{\partial #2}}
  If $z = x^2 + xy + y^2$, then
  \[
    \der{z}{x} = 2x + y
  \]
\end{verbatim}
Output:

  \newcommand{\pder}[2]{\frac{\partial #1}{\partial #2}}
  If $z = x^2 + xy + y^2$, then
  \[
    \pder{z}{x} = 2x + y
  \]
\end{document}
