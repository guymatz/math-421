\documentclass{article} % This command is used to set the type of document you are working on such as an article, book, or presenation

%\usepackage[margin=1in]{geometry} % This package allows the editing of the page layout. I've set the margins to be 1 inch. 

\usepackage{amsmath, amsfonts}  % The first package allows the use of a large range of mathematical formula, commands, and symbols.  The second gives some useful mathematical fonts.

\usepackage{graphicx}  % This package allows the importing of images
\usepackage{marvosym}  % Lightning!

%This allows us to use the theorem and proof environment 
\usepackage{enumitem}
\usepackage{amsthm}
\theoremstyle{plain}
\newtheorem*{theorem*}{Theorem}
\newtheorem{theorem}{Theorem}

\newtheoremstyle{case}{}{}{}{}{}{:}{ }{}
\theoremstyle{case}
\newtheorem{case}{Case}

%Custom commands.  
\newcommand{\abs}[1]{\left\lvert #1 \right\rvert} %absolute value command

%Custom symbols
\newcommand{\Rb}{\mathbb{R}}

\begin{document}

\begin{center}
\Large{\textbf{Assignment \#2}
            
UW-Madison MATH 421} % Name of course here
\vspace{5pt}
        
\normalsize{  Guy Matz% Your name here
        \\ Due: June 25, 2023}
\vspace{15pt}
\end{center}

\section*{Exercises}%
\begin{enumerate}[label={\fbox{\textbf{Exercise \#\arabic* :}}}]
\item Prove the following theorem by cases. 

\begin{theorem*}
  If $x$ is an integer, then $x^2 + 3x-9$ is odd. 
\end{theorem*}

$P = x$ is an integer\\
$Q = x^2 + 3x - 9$ is odd

\begin{proof}
  \begin{case}
    x is even

    If we assume that x is even, then there exists a $k$ such that $x=2k$.
    Then the equation above becomes $(2k)^2 + 3\dot 2k -9$, which is equal
    to $4k^2 + 6k - 9$.  Factoring out 2 we get $2(2k^2 + 3k - 5) + 1$.
    Now let $w = 2k^2 + 3k -5$ and we have $2w + 1$, which is an odd
    number
  \end{case}
  \begin{case}
    x is odd

    If we assume that x is odd, then there exists a $k$ such that $x=2k+1$.
    Then the equation above becomes $(2k+1)^2+3\dot(2k+1)-9$, which is equal
    to $4k^2 + 10k - 5$.  Factoring out 2 we get $2(2k^2 + 5k - 3) + 1$.
    Now let $w = 2k^2 + 5k -3$ and we have $2w + 1$, which is an odd
    number
  \end{case}
\end{proof}

\newpage
\item Prove the following theorem in two ways: by contrapositive and by contradiction. 

\begin{theorem*}
  Suppose $x$ is an integer. If $x^2$ is even, then $x$ is even. 
\end{theorem*}

$P = x^2$ is an even integer\\
$Q = x$ is even

\begin{proof}[\textbf{Proof by contrapositive}]  We want to show that if
  $x$ is not even, then $x^2$ is not even.
  I.e, $\neg Q \Rightarrow \neg  P$

  If $x$ is not even, then it is odd and can be expressed as $x = 2k+1$
  for some integer $k$.  $x^2$ then, is $(2k+1)^2$, or $4k^2 + 4k + 1$.
  Let $w = 2(2k^2 + 2k) + 1$, then $w$ is odd and we have shown that if
  $x$ is not even, then $x^2$ is not even. 

\end{proof} 

\renewcommand\qedsymbol{\Lightning}
\begin{proof}[\textbf{Proof by contradiction}]  Towards a contradiction,
  Let's assume that $x$ is odd and $x^2$ is even.  We want to show this
  leads to a contradiction.  I.e. $\neg Q \wedge P \Rightarrow \neg P$

  Since $x$ is odd we can find an integer $k$ such that $x = 2k + 1$. Then 
  $x^2 = 4k^2 + 4k + 1$, which is equal to $2(k^2 + 2k) + 1$, but that's
  an odd number and we said that $x^2$ is even!
\end{proof} 
\renewcommand\qedsymbol{$\square$}

\newpage
\item Prove the following theorem. 

\begin{theorem*}
For all numbers $x$ and $y$,  $(x+y)^2=x^2+y^2$ if and only if $x=0$ or $y=0$. 
\end{theorem*}

P:$ (x+y)^2=x^2+y^2$ \\
Q:$ x=0 \text{ or } y=0$

\begin{proof}
~\\
  ($\Rightarrow$): First we want to show  if $(x+y)^2 = x^2 + y^2$, then
  either $x=0$ or $y=0$.  I.e $P \Rightarrow Q$

  Expanding $(x+y)^2$, we have that $x^2 + 2xy + y^2 = x^2 + y^2$ which leads us to
  $2xy = 0$.  This only only soluble when either $x$ or $y$ is equal to 0. 

  $(\Longleftarrow)$: Next we want to show if either $x=0$ or $y=0$, then
  $(x+y)^2 = x^2 + y^2$.  I.e $Q \Rightarrow P$

   Without loss of generality, let's choose $x=0$.  Then
   $(x+y)^2 = x^2 + y^2$ becomes $x^2 + 2xy + y^2 = x^2 + y^2$,
   and with $x=0$ we have $y^2 = y^2$, which is obviously true.
\end{proof}

\newpage
\item Using only the results established in class and properties P1-P12
  (and noting every time you use one), prove the following theorem. 

\begin{theorem*}
Suppose $a$ and $b$ are numbers. If $ab=1$, then $b = a^{-1}$. 
\end{theorem*}
\textit{Hint: since $ab \neq 0$ a theorem in class implies that $a \neq 0$. Which one?}

P:$ a \cdot b = 1$ \\
Q:$ b = a ^{-1}$

\begin{proof}
  By T4 we have that neither $a=0$ nor $b=0$.  We then know that a
  multiplicative inverse exists (P7), so we can pre-multiply both sides
  of $a \cdot b = 1$ by $a ^{-1}$ and we get
    $$a ^{-1} \cdot a \cdot b = a ^{-1} \cdot 1$$
  Then we have
  \[ b = a ^{-1} \cdot 1 \tag*{(P7)} \]
  And finally
  \[ b = a ^{-1} \tag*{(P6)} \]
\end{proof} 

\newpage
\item Using only the results established in class and properties P1-P12 (and noting every time you use one), prove the following theorem. 

\begin{theorem*}
Suppose $a$ and $b$ are numbers. If $a \neq 0$ and $b \neq 0$, then $(a b)^{-1} = a^{-1} b^{-1}$. 
\end{theorem*}
\emph{Hint: you can use the previous problem.}

P: $a \neq 0$ and $b \neq 0$\\
Q: $(a b)^{-1} = a^{-1} b^{-1}$. 

\begin{proof}
If $a \neq 0$ and $b \neq 0$, then $(a \cdot b) \neq 0 $
(Contrapositive of T4).  And multiplying $a ^{-1} \cdot b ^{-1}$ by
$(a \cdot b)$ will give us 1 (P7).
  \begin{align*}
    (a ^{-1} \cdot b ^{-1} ) \cdot (a \cdot b) & = (a ^{-1} \cdot b ^{-1} ) \cdot (b \cdot a)\tag*{(P8)} \\
      & = (a ^{-1} \cdot b ^{-1} \cdot b ) \cdot (a)\tag*{(P5)} \\
      & = (a ^{-1} \cdot 1 ) \cdot (a)\tag*{(P7)} \\
      & = (a ^{-1} ) \cdot (a)\tag*{(P6)} \\
      & = (a ^{-1} \cdot a)\tag*{(P5)} \\
      & = (1)\tag*{(P7)} \\
  \end{align*}

\end{proof} 

\newpage
\item Using only the results established in class and properties P1-P12 (and noting every time you use one), prove the following theorem. 

\begin{theorem*}
Suppose $a$, $b$, and $c$ are numbers. If $a<b$ and $0<c$, then $ac < bc$. 
\end{theorem*}

P: $a<b$ and $0<c$\\
Q: $ac < bc$

\begin{proof}
  Because $a < b$, we know that $b - a \in \mathbb{P}  $.  And,
  since $0 < c$, we can say that $c \in \mathbb{P}$.
  So $c(b-a) \in \mathbb{P}  $ (P12).  Then
    \begin{align}
      c(b-a) &= cb-ca \tag*{(P9)} \\
             &= bc -ac \tag*{(P8)}
    \end{align}
    And we have that $bc - ac \in \mathbb{P}$. Hence $ac < bc$ (I2)
\end{proof} 

\newpage
\item Prove the following: if $x$ and $y$ are numbers, then 
\begin{enumerate}[label=(\arabic*)]
\item $|x-y| \leq |x|+|y|$
\item $|x|-|y| \leq |x-y|$
\end{enumerate}
\textit{Hint: you can give a short proof of both facts by reducing to the
  triangle inequality.  You do not have to reference properties P1-P12.}

P: ${x, y} \in \mathbb{R}$\\
Q: $|x-y| \leq |x|+|y|$

\begin{proof}[Proof of (1)]
  $|x-y| \leq |x|+|y|$ can be rewritten as $|x+z| \leq |x|+|(-z)|$, where
  $z = -y$.  This is the Triangle Inequality.
\end{proof}

P: ${x, y} \in \mathbb{R}$\\
Q: $|x|-|y| \leq |x-y|$

\begin{proof}[Proof of (2)] We can rewrite the equation as the 
  Triangle Inequality, first letting $x = x + y$.
  \begin{align*}
    |x| - |y| &\leq |x-y| \\
    |x + y | - | y | &\leq |x + y - y| \\
    |x + y| &\leq |x + y - y | + |y| \\
    |x + y| &\leq |x | + |y| \\
  \end{align*}
\end{proof}

\newpage
\item \textbf{(Optional)} The minimum of two numbers $x$ and $y$ are
  denoted by $min (x,y)$.

  For example, $min(-3, 1)=-3$ and $min (2,3)=2$.

    Prove the following statement: If $y$ is a number such that $y\neq 0$ and
    \[ |x-y| < min \left(\frac{|y|}{2}, \frac{\epsilon\cdot |y|^2}{2}\right) \]
    for some positive number $\epsilon \neq 0$.
    Then prove that $x\neq 0$ and

    \[ \left|\frac{1}{x}-\frac{1}{y}\right| <\epsilon. \]

    P: $y \neq 0$, and 
    \[ |x-y| < min \left(\frac{|y|}{2}, \frac{\epsilon\cdot |y|^2}{2}\right) \]
    Q: $x \neq 0$, and
    \[ \left|\frac{1}{x}-\frac{1}{y}\right| <\epsilon. \]

      \begin{proof}[\textbf{Proof by Contrapositive}]
        There are two cases here, however to show the contrapositive we
        need only show one.  (I think I might be wrong here!)
        We will show that
            If $x = 0$, then $|x-y| > min(\frac{|y|}{2} , \frac{e\dot |y|^2}{2})$

                Surely this is true, since, if $x = 0$, the equation becomes
                \[ |-y| > min(\frac{|y|}{2} , \frac{e\dot |y|^2}{2}) \]
                and surely $y$ is greater than $\frac{y}{2}$, so $y$
                is greater than one of $\frac{|y|}{2},
                \frac{\epsilon|y|^2}{2}$, so it is greater than the
                minimum.
      \end{proof} 

\end{enumerate}

\newpage
\section*{Theorems}

    Suppose $a, b, c, x$ are real numbers,:
    \begin{theorem}
          $a+x=a \Rightarrow x=0$
    \end{theorem}
    \begin{theorem}
        If $a \cdot b=a \cdot c$ and $a \neq 0$, then $b = c$
    \end{theorem}
    \begin{theorem}
        $a \cdot 0 = 0$ and $0 \cdot a = 0$
    \end{theorem}
    \begin{theorem}
        If $a \cdot b=0$, then either $a=0$ or $b=0$
    \end{theorem}

    \begin{theorem}[Triangle Inequality]

    If $x$  and $y$ are real numbers, then
    \[ |x+y| \leq |x| + |y| \]
    \end{theorem}
  \begin{proof}
    \begin{case}
      $x \geq 0, y \geq 0$

      Then $|x+y| = x+y$ (Trichototmy for P).  $|x| = x, |y|=y \Rightarrow |x|+|y| = x+y$

      So, $|x+y| = x+y = |x|+|y| \leq |x|+|y|$
    \end{case}
    \begin{case}
      $x \leq 0, y \leq 0$

      Then $|x+y| = -(x+y)$o
      \[ x \leq 0 \Rightarrow  -x \in \mathbb{R}^+ \]
      \[ y \leq 0 \Rightarrow  -y \in \mathbb{R}^+ \]
      so $-x-y = -(x+y) \in \mathbb{R}^+$

      $|x| = -x, |y|=-y$ since $x \leq 0, y \leq 0$

      So, $|x|+|y| = -x + -y = -(x+y)$

      \[ |x+y| = -(x+y) = |x| + |y| \leq |x| + |y| \]
    \end{case}
    \begin{case}
      Assume $x \geq 0, y \leq 0$.  Now, $|x| = x, |y| = -y$
      \begin{proof}
        \begin{case}
          Assume that $x+y \geq 0$.  By assumption $|x+y| = x+y$

          \vdots
        \end{case}
        \begin{case}
          second

        \end{case}
      \end{proof}

    \end{case}
    Case 4 : $ x \leq 0, y \geq 0$

    \[ |x+y|=|y+x| (P4) \leq |y|+|x| \text{(by case 3)} = |x| + |y| P(4)\]
  \end{proof}


\section*{Properties}
  \begin{enumerate}
        \item[P1]: Additive Associativity - $a + (b+c) = (a+b)+c$
        \item[P2]: Additive Identity - $a + 0 = 0+a = a$
        \item[P3]: Additive Inverse-For all real number $a$ there exists
          an $-a$ such that
            $$a + -a = 0$$
        \item[P4]: Commutativity of Addition - $a+b = b+a$ or all real
          numbers $a,  b$
        \item[P5]: Associativity for Multiplication
          \[ a \cdot (b \cdot c) = (a \cdot b) \cdot c \]
        \item[P6]: Identity for Multiplication - $a \cdot 1 = 1 \cdot a = a$
        \item[P7]: Multiplicative Inverse - Suppose $a \neq 0$, thn
          there exists a real number $a^{-1}$ such that
          $a \cdot a^{-1} = a^{-1} \cdot a = 1$
        \item[P8]: Commutativity for Multiplication - $a \cdot b=b \cdot a$
          for any two real number $a,b$
        \item[P9]: Distributivity for Multiplication -
          $a \cdot (b +c) = a \cdot b  + a \cdot c$
          for any real numbers $a,b,c$
        \item[P10] For every real number $a$ , exactly one of the
          following holds
          \begin{enumerate}
            \item $a=0$
            \item $a \in P$
            \item $-a \in P$
          \end{enumerate}
        \item[P11]: If $a,b \in P$ then $a+b \in P$
        \item[P12] If $a,b \in P$ then $a \cdot b \in P$
  \end{enumerate}

\section*{Properties of Inequalities}
    \begin{enumerate}
      \item $ a> b$ if $a-b \in P$
      \item $ a < b$ if $b > a$ ($b-a \in P)$
      \item $ a \geq b$ if $a > b$ or $a = b$
      \item $ a \leq b$ if $b \geq a$
    \end{enumerate}

\end{document}
