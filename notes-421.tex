\documentclass{report}

\input{.tex/preamble}
\input{.tex/macros}
\input{.tex/letterfonts}

\title{
  \Huge{Math 421---The Theory of Single Variable Calculus}
  \\
  Notes
}
\author{\huge{Guy Matz}}
\date{}
\begin{document}

\section*{20230619 - Intro to Math Arguments}%
  \begin{itemize}
          \item Stamements \& Locgivasl Operations

            Statments are sentinves which are eithertrue of false (not both)

            e.g. The number 6 is even

            \underline{OPERATIONS on statements}

            P \& Q are staements.  We can modify and compnine these etatements in
            difffferent ways.

            NOT: $\neg$ - The statement "$\neg P$" is true when P is false

            AND $\wedge$ - The statement "$P \wedge Q $" is true when both P and Q

            OR $\vee$ - The statement "$P \vee Q $" is true when at least one of
            P and Q is true.  False when both P \& Q are false

            IF ... THEN $\Longrightarrow $: The Statement "If P, then Q" is true when
            \begin{itemize}
              \item P is true and Q is true
              \item P is False
            \end{itemize}

            False when P is true and Q is false

            $P \Longrightarrow Q$ is equivalent to $(P \wedge Q) \vee (\neg P)$

            IF AND NLY IF\\
            P iff Q ($P \Leftrightarrow Q$ is true when either
           \begin{itemize}
              \item P and Q are both true
              \item P and Q are both false
            \end{itemize}

            \underline{FOR P $\Longrightarrow Q$ we have the following}

            CONVERSE: The converse of is $Q \Longrightarrow P$
            CONTRAPOSITIVE The constrspoasitive is $\neg Q \Longrightarrow \neg P$

            $P \Longrightarrow Q$ is logically equivalent to its contrapositive $(\neg Q \Longrightarrow \neg P)$


        two statements are logivally equivalent if they have the same truth tables.

        \underline{WRITE OUT TRUTH TABLES}

        Compare the truth tables for if..then, contrapositive and $(P \wedge Q) \vee (\neg P)$

        P    Q   $P \wedge Q)$ $\neg P$      $(P \wedge Q) \vee \neg P$ \\
        T    F \\
        T    T \\
        F    T \\
        F    F \\

        \underline{DEFINITIONS}
        We say an integer n is even if there exists an integer k such that $n = 2k$

        We say an integer n is odd if there exists an integer k such that $n = 2k+1$

        \thm{ $x^2 + 1$ is even } {
          Suppose x is an integer.  If x is odd, then $x^2 + 1$ is even.
        }
        \myproof {
          Since xx is odd, $x = 2k+1$ ffor some integer k.  The $x^2 + 1 = (2k+1)^2 + 1) = 4k^2 + 2k + 2) = 2(k^2 + k + 1)$.  Since k is an integer, $sk^2  + 2k =1$ is also
          an integer.  So, $x^2 + 1 = 2m$, where $m = 2k^2 + 2k + 1$.
          Hence we are done
        }
        \thm{ x is even } {
          For every integer x, x is even iff $x + 1$ is odd
        }
        \myproof {
          $\Longleftarrow$ First we want to show $x$ is even $\Longrightarrow x+1$ is odd

          Since x is even, $x = 2k$ for some integer k.  So, $x+1 = 2k+1$.  Then by  defn, $x+1$ is odd

          $\Longrightarrow$ Next we want to show $x+1$ is odd $\Longrightarrow $ x is even.

          If x+1 is odd, $x+1 = sk+1$ ffor some integer k.  This measnt that $x = 2k$,
          By defn, x is even
        }

        REMARK: Proof by cases

  \end{itemize}

\section*{20230620 - Proof Techniques}%
  \begin{itemize}
    \item Theorems are off the for $P \Longrightarrow Q$
    \item Proof by Cases

      Divide P into multiple cases

    \item Proof by Contrapositive

      $P \Longrightarrow Q $ is equivalent to $\neg Q \Longrightarrow \neg P$
      Start the proof by stating "Proof by contrapositive: Assume $\neg Q$
      and we will show that $\neg P$ is true

    \item Proof by Contradiction

      Assume that Q is not true.  Then we want to reach a contradiction
        \[ \neg Q \wedge P \Longrightarrow \neg P \]
      Start the proof by saying "Proof by contradiction:  Assume that
      Q is not true ... "  or "Assume for a contradiction that Q is not
      true"
  \end{itemize}

  \subsection*{Examples}%
  \ex{} {
    \thm{ $x + y \geq 19$ } {
      Assume that x,y are integers.  Then $x+y > 19$ implies that
          $x \geq 10$ or $y \geq 10$
    }
    P is $x + y \geq 19$ and `x and y are integers.  Q is $x \geq 10 \vee
    y \geq 10$
    \myproof {
        Case 1: $x \geq 10$.  Then you are done, as the conclusion is satisied

        Case 2: $x < 10$.  By assumption, $x + y \geq 19$. Then
        $x + y \geq 19 \Longrightarrow y \geq 19-x$.

        Since $x < 10$, $19 -x \geq 19-9 \geq 10$.  Now $ y \geq 19-x
        \Longrightarrow y \geq 10$
    }

    \myproof {
      Proof by Contrapositive:
        \[ \neg (x + y \geq 19 ) == x + y < 19 \]
        \[ \neg (x \geq 10 \vee y \geq 10) == x < 10 \wedge y < 10 \]

        Assume that $x < 10 , y < 10$.  Since x and y are integers,
        $x \leq 9$ and $y \leq 9$.  So $x + y \leq 9 + 9 = 18$.  Hence,
        $x + y < 19$
      }

      \renewcommand\qedsymbol{\Lightning}
      \begin{proof}[\textbf{Proof by contradiction}]  Towards a contradiction,
        Assume THAT $ x \leq 9$ and $y \leq 9$.  Also we have that
        $x + y \geq 19$.  Since $x \leq 9$ and $y \leq 9$,
        $x + y \leq 9 + 9 = 18$.  But we know that $x + y \geq 19$.  But
        this is a contradiction.
      \end{proof}
      \renewcommand\qedsymbol{$\square$}
  }

  \subsection*{Set Theory}%
    \dfn{ Set }{
      A set is a collection of unordered objects
    }
    Let $\mathbb{R}$ be the real numbers.
    \begin{enumerate}
      \item If $a < b, (a,b)$ is the set $(a,b) := \{x: x \in \mathbb{R} , a < x < b \}$
      \item $[a,b) := \{x | x \in \mathbb{R} , a \leq x < b \} $
      \item Empty set is the set that contains no elements: $\emptyset$
      \item Since a set is unordered, $\{1,2,3\}$ is the same as $\{2,3,1\}$
      \item Since we don't consider duplicates, $\{1,2,2,3\ = \{1,2,3\}$
    \end{enumerate}

    \underline{SUBSETS}
    \begin{itemize}
      \item Let S be a set.  Then we say that all of the  elemtns of B is a subset of A if all the elements of B are sslso elements of A
        . $ B \subseteq A$
      \item $B = \{x | x \in A, \text{some condition on x} \}$
      \item $B = \{x | x \in \mathbb{R}, x = \frac{1}{n}, \text{for some integer} n $
    \end{itemize}

    \underline{OPERATIONS on SETS}
    \begin{itemize}
      \item Intersection: $\cap$.  E.g. $(1,3) \cap (2,4) = (2,3)$
      \item Union: $\cup$  E.g. $(1,3) \cup (2,4) = (1,4)$
      \item Product: $A \times B = \{(a,b) | a \in A, b \in B \}$
    \end{itemize}

    \underline{Practice Problems}
    \begin{itemize}
      \item Show that $\neg(P \vee Q) = \neg P \wedge  \neg Q$ by using
      \item Show that $\neg(P \wedge Q) = \neg P \vee  \neg Q$ by using
        a truth table
      \item Let $x$ be an integer.  Then  $x$ is even iff $\frac{x}{2} $ is an integer
      \item Let $a,b$ be integers.  Then $ab \geq 9$ implies $a \geq 3$ or
        $ b b\geq 3$
        \begin{enumerate}
          \item What is $P, \neg P, \neg Q$
          \item Prove this using all three techniques
        \end{enumerate}
    \end{itemize}


\section*{20230621 - More Proof Techniques}%
  \subsection*{Review}%
  \begin{itemize}
    \item 3 proof techniques
    \item Basic Set theory
  \end{itemize}

  \subsection*{Real Numbers}%
    \dfn{ Real Numbers }{
      The real numbers denoted by $\mathbb{R}$ is a set with the
      following algebraic properties
      \begin{itemize}
      \item Two operations, $\{+ , \cdot\}$
      \item Two special numbers beloinging to $\mathbb{R}$, $\{0,1\}$
      \item The set of positive integers $(P), P \subset \mathbb{R}$
      \item 13 Properties
      \end{itemize}
    }

    \underline{Remark}
    \begin{enumerate}
      \item It is not clear iff such a set exists
      \item If the set exists, it is not clear iff it is unique
      \item 13th property is hard to understand
    \end{enumerate}

    \underline{SOLUTION!}
    \begin{enumerate}
      \item Ignore remark 1 and 2 above
      \item Delay discussing 13th property (Least Upper Bound)
    \end{enumerate}

    \underline{PLAN FOR CHAPTER 1}
    \begin{enumerate}
      \item Discuss the 1st 12 properties of real numbers
      \item addition, subtraction, inequalities
      \item prove more algebrai properties of $\mathbb{R}$ using the
        first 12 properties
    \end{enumerate}

    \subsection*{The First 4 Properties (for addition)}%
      \begin{enumerate}
        \item[P1]: Additive Associativyy - $a + (b+c) = (a+b)+c$
        \item[P2]: Additive Identity - $a + 0 = 0+a = a$
        \item[P3]: Additive Inverse-For all real number $a$ there exists
          a $-a$ such that $a + -a = 0$
        \item[P4]: Commutativity of Additive - $a+b = b+a$ or all real
          numbers $a,  b$
      \end{enumerate}

      \ex{} {
        \thm{ 1 } {
          Suppose $a$ and $x$ are real numbers, then $a+x=a \Rightarrow x=0$
        }
        \myproof {
          Add $-a$ (P3) to both sides of the equation.  Then
          $-a+(a+x) = -a + a = 0$ by P3.  Now,
          $-a+(a+x) = (-a+a)+x) (P1) = 0 + x (defn) = x (P2)$.
          By comparing (1) and (2) you get x=0
        }

        \myproof {
          Suppose that $a+x=a$.  We know that
          0 = -a + a (P3) \\
          = -a + (a+x) by assupmption\\
          =(-a+a)+x (P1)\\
          = 0 +x (P1)\\
          = x (P2)
        }
      }

    \dfn{ Subbtraction }{
      Using P3 we define subtraction For real numbers $a,b$

      We define $a-b=a+-b$
    }

    \subsection*{The Properties for Multiplication}%
      \begin{enumerate}
        \item[P5]: Associativy - $a + (b+c) = (a+b)+c$
        \item[P6]: Identity - $a \cdot 1 = 1 \cdot a = a$
        \item[P7]: Multiplicative Inverse - Suppose $a \neq 0$, thn
          there exists a real number $a^{-1}$ such that
          $a \cdot a^{-1} = a^{-1} \cdot a = 1$
        \item[P8]: Commutativity - $a \cdot b = b \cdot$
          for any two real number $a,b$

      \thm{ 2 } {
        Suppose $a,b,c$ are real numbers such that {**}$a \cdot b=a \cdot c$
        and $a \neq 0$, then $b = c$
      }
      \myproof {
        $a^{-1}$ exists by P7. Pre-multiply $a^{-1}$ to both sides of **
        .  So $a^{-1} (a \cdot) = a^{-1}(ac)$. By P5,
        $a^{-1}(ab) = (a^{-1}a)b$ and  $a^{-1}(ac) = (a^{-1}a)c$.  By
        P7 $a^{-1}a = 1$
      }

      \dfn{ Division }{
        Using multiplicative inverse, suppose $b \neq 0$, then $a/b = a \cdot b^{-1}$
      }
        \item[P9]: Distributivity -
          $a \cdot (b +c) = a \cdot b  + a \cdot c$
          for any real numbers $a,b,c$

          \thm{ 3 } {
            Suppose $a$ is a real number, then $a \cdot 0 = 0$ and
            $0 \cdot a = 0$
          }

          \myproof {
             Consider $a 0 + a0$.  By P9 ***$a0 + a0 = a(0 + 0)$.  By
             P2 $0 +0 =0$.  Then *** becomes $a0+a0=a0$.  By Theroem 1
             (above) $a0=0$
          }
      \thm{ 4 } {
        Suppose a,b are real numbers st $ab=0$, then either a=0 or b=0
      }
      \begin{proof}
        \begin{case}
          Suppose a = 0, then we are done
        \end{case}
        \begin{case}
          Suppose $a \neq 0$, then P7 gives $a^{-`}$.
          By premultiplying $a^{-1}$ to the eqn
          $a^{-1}(a b) = a^{-1}(ab) = a ^{-1}0.
          By P5 a index(ab) = (a ^{-1} a) b$.  By P7 $a ^{-1} a = 1$.
          By Thm 3 $a ^{-1} 0 = 0$.  By P6 $a b = b$, so $b=0$

        \end{case}
      \end{proof}

    \underline{3 Properties of Positive Numbers}
    \item[P10] For every real number $a$ , exactly one of the following
      holds
      \begin{enumerate}
        \item $a=0$
        \item $a \in P$
        \item $-a \in P$
      \end{enumerate}
    \item[P11]: If $a,b \in P$ then $a+b \in P$
    \item[P12] If $a,b \in P$ then $a \cdot b \in P$
  \end{enumerate}

  P10 - P11 allows us to define inequalities
  \dfn{ InequLITIES }{
    \begin{enumerate}
      \item $ a> b$ if $a-b \in P$
      \item $ a < b$ if $b > a$ ($b-a \in P)$
      \item $ a \geq b$ if $a > b$ or $a = b$
      \item $ a \leq b$ if $b \geq a$
    \end{enumerate}
  }
  \thm{ 4 } {
    If $a,b,c$ are R such that $a < b$ and $b < c$, then $a < c$
    Transitivity of <
  }
  \myproof {
    $ a < b$ means $b-a \in P$
    $ b < c$ means $c-b \in P$

    By P11, $(b-a + c-b \in P$

    By P4, $(b-a + c-b = (b-a) + (-b+c)$

    By P1, $(b-a + -b + c = (b-a + -b) +c$

    By P4, $(b-a + -b) + c = (b-b-a) +c$

    By P3, = $(0-a) + c$

    By P2 $= -a + c$

    So $(b-a) + (c-b) \in P$, which is the same as $c-a \in P$ by P4
  }





\section*{20230622 - Absolute Values}%
  \subsection*{Recap}
    \begin{itemize}
      \item Poperties P1-P12
      \item Inequalities
    \end{itemize}

  \subsection*{Absolute Values}%
  \dfn{ Absolute Value }{
    Let a be a real number.  We deffine
    \[ |a| = a \text{ if } a \geq 0 \]
    \[ |-a| = a \text{ if } a \leq 0 \]
  }
  \ex{} {
    \[ |7| = 7 \]
    \[ |-7| = 7 \]
  }
  \textbf{Remark}: It is helpful to deal with cases for questions related to absolute values

  \subsection*{Triangle Inequality}%
  \thm{ Triangle Inequality} {
    If $x$  and $y$ are real numbers, then
    \[ |x+y| \leq |x| + |y| \]
  }
  \begin{proof}
    \begin{case}
      $x \geq 0, y \geq 0$

      Then $|x+y| = x+y$ (Trichototmy for P).  $|x| = x, |y|=y \Rightarrow |x|+|y| = x+y$

      So, $|x+y| = x+y = |x|+|y| \leq |x|+|y|$
    \end{case}
    \begin{case}
      $x \leq 0, y \leq 0$

      Then $|x+y| = -(x+y)$o
      \[ x \leq 0 \Rightarrow  -x \in \mathbb{R}^+ \]
      \[ y \leq 0 \Rightarrow  -y \in \mathbb{R}^+ \]
      so $-x-y = -(x+y) \in \mathbb{R}^+$

      $|x| = -x, |y|=-y$ since $x \leq 0, y \leq 0$

      So, $|x|+|y| = -x + -y = -(x+y)$

      \[ |x+y| = -(x+y) = |x| + |y| \leq |x| + |y| \]
    \end{case}
    \begin{case}
      Assume $x \geq 0, y \leq 0$.  Now, $|x| = x, |y| = -y$
      \begin{proof}
        \begin{case}
          Assume that $x+y \geq 0$.  By assumption $|x+y| = x+y$

          \vdots
        \end{case}
        \begin{case}
          second

        \end{case}
      \end{proof}

    \end{case}
    Case 4 : $ x \leq 0, y \geq 0$

    \[ |x+y|=|y+x| (P4) \leq |y|+|x| \text{(by case 3)} = |x| + |y| P(4)\]
  \end{proof}

  \thm{ $y \leq 0  \leq -y$ } {
    If $ y \leq 0$, then  $y \leq 0  \leq -y$
  }

  \ex{} {
    If $a$ is a real number, then
    \[ (-1) a = -a \]

    \begin{enumerate}
      \item Add $a$  to both sides to get $a + (-1) a = a + -a$ = 0 (P2)
      \item $a = a \cdot 1$ (Identity mult)
      \item $(-1) \cdot a = a\cdot (-1)$ (Commutativity
      \item $a+(-1) \cdot a = a(1 + -1) = a \cdot 0 = 0$
    \end{enumerate}
  }

  \ex{} {
    If $x$ is a real number
    \[ |-x| = |x| \]

    Use abs defn to show $|-x| = |x|$
  }

  \ex{} {
    If $x,y$ are reals, then
      \[ |xy| = |x||y| \]

      4 cases
      \begin{enumerate}
        \item $|x| \cdot |y| = xy$ if $x \geq 0, y \geq 0$
        \item $|x| \cdot |y| = x(-)y$ if $x \geq 0, y \leq 0$
        \item $|x| \cdot |y| = -x \cdot -y$ if $x \leq 0, y \leq 0$
        \item $|x| \cdot |y| = (-x)y$ if $x \leq 0, y \geq 0$
      \end{enumerate}
      equals
      \begin{enumerate}
        \item $xy$ if $x \geq 0, y \geq 0$, etc
        \item $-xy$ if $x \geq 0, y \leq 0$,  etc
      \end{enumerate}
  }

  \subsection*{Chapter 2 - Numbers of Various Sorts}%
  \subsubsection*{Notation}%
    \begin{itemize}
      \item $\mathbb{N} $ : Naturals
      \item $\mathbb{Z} $ Integers
      \item $\mathbb{Q} $ RationalsA
      \item $\mathbb{R}$ : Reals
    \end{itemize}

    \[ \mathbb{N}   \subset \mathbb{Z} \subset \mathbb{Q} \subset \mathbb{R} \]

    Does $\mathbb{N}  $ saitisfy P1 - P12?  No, missing 2, 3, \& 7

    Does $\mathbb{Z}  $ saitisfy P1 - P12?  No, missing 7

    Does $\mathbb{Q}  $ saitisfy P1 - P12?  YES

    \underline{NOTE} The preoperty that distinguishes $\mathbb{Q} $ from
    $\mathbb{R}$ is the 13th property

    \underline{NOTE}:  Assume that sqrt(2) exists.  I.e. we can
    find a solution to $x^2 = 2$

    \thm{$\sqrt{2}$ is irrational}{
      Suppose for a contradiction that $\sqrt{2}$ is rational.
      Let $\sqrt{2} = \frac{p}{q} $.  By  cancelling out commen factors
      from p and q, we can assume p and q not both even.  Then
      \[ \sqrt{2} = \frac{p}{q} \Rightarrow 2 = \frac{p^2}{q^2}   \]
      Multiply $q^2$ on both saides to get $2q^2 = p^2$
    }

    \underline{Plan for Chapter 2}
    \begin{itemize}
      \item Different numbers
      \item $\sqrt{2}$ is irrational
      \item Proof technique - induction
    \end{itemize}
\section*{20230626 - Induction}%
  \thm{ $x+y$ is irrational } {
    If $x$ is irratinoal and $y$ is rational, $x+y$ is irrational
  }
  \renewcommand\qedsymbol{\Lightning}
  \begin{proof}[\textbf{Proof by contradiction}]  
    So, if $x+y$ is ratinal, then $x+y = 2 \Rightarrow $ rational  implies $x = 2-y$
  \end{proof} 
  \renewcommand\qedsymbol{$\square$}

  \thm{ Well-Ordering Principle } {
    Every non-empty subset of natural numbers has a least (smallest)
    element
  }

  \ex{NON-example!} {
    $\mathbb{Z} , \mathbb{Q} and \mathbb{R}$ do not have this property
    \begin{enumerate}
      \item $S = \{ \dots, -2, -1 , 0, 1, \dots \} \subset \mathbb{Z} , \mathbb{Q} , \mathbb{R}$
    \end{enumerate}
  }

  \dfn{ Principle of Mathematical Induction }{
    Let $P(n)$ be a statement that depends only on the natural number $n$.
    Suppose the ollowing holds
    \begin{itemize}
      \item Base Case --- $P(1)$ is true
      \item Induction Step --- If $P(k)$ is true for some natural
        number, $k$, then $P(k+1)$ is also true.
    \end{itemize}
      Then $P(n)$ is true for all natural numbers
  }
  \renewcommand\qedsymbol{\Lightning}
  \begin{proof}[\textbf{Proof by contradiction}]  Towards a contradiction,
    Let's assume that $P(N)$ is not true for all $n$.  Then the set
    \[ S = \{n:P(n)\} \text{ is not true } \]
    is non-empty.  So, by well-ordering principle we have a least element,
    say $k$, in $S$.  Siince $k$ is the least element in $S, k-1$ does not
    belong to $S$.  But this implies that $k-1$ is true.  By induction
    step, $P(k-1+1) = P(k)$ is true.  This implies that $k$ does not
    beloong to $S$.  This is a contradiction.
  \end{proof} 
  \renewcommand\qedsymbol{$\square$}

  \ex{Prove by Indutcion} {
    $\sum^{n}_{i=1} i = \frac{n(n+1)}{2} $ for all $n \in \mathbb{N}$
    \myproof {
      \underline{Base Case: }
      \[ \sum^{1}_{i=1} 1 \] 
      \[ \frac{1(1 + 1)}{2 = 1}  \]
      \underline{Induction Step }
      \[ \text{ Assume }\sum^{k}_{i=1} \frac{k(k+1)}{2}  \] 
      So, .... do this!
    }
  }

%  \nt{ Induction is widely used, so we usually drop stating $P(n)$, i.e.
%  The Induction Step}


    \thm{ title } {
      Let $P(n)$ be a statement that depends only on $\mathbb{N}$.  Then
      the following holds
      \begin{enumerate}
        \item $P(N_0)$ is true for some $n_0 \in \mathbb{N}  $
          $P(k)$ is true for some $k \geq n_0 \implies P(k+1)$ is true
      \end{enumerate}
      Then $P(n)$ is true for all.  I.e., the base case could be any
      natural number
    }

    \myproof {
       Consider the statement $Q(n) := P(n + n_0 -1$.

      \underline{Base Case: } $Q(1) = P(1 + n_0 -1) = P(n_0)$

      \underline{Induction Step: } Assume $Q(k)$ is true for some
      $k \in \mathbb{N}$, i.e. $P(k + n_0 -1)$ is true.

      The assumption on $P$ gives us that $P(k + n_0 -1 + 1) = P(k + n_)
       = Q(k + 1)$ is true.

       Therefore by Induction Principle, $Q(n)$ is true for all $n \in 
       \mathbb{N}  $ i.e. $P(n)$ is true for all $n \geq n_0$
    }
    
    \ex{$n! \leq n^n$ } {
      Using Induction, show that $n! \leq n^n$ for all $n \geq 1$
      \myproof {
        \underline{Base Case: } $1 \leq 1^1$
        \underline{Induction Step } Assuming $n! \leq n^n$, we want to show
        $(n+1)! \leq (n+1)^{n+1}$. $(n+1)! = (n+1) \cdot n!$.  And 
        $(n+1)^{n+1} = (n+1) \cdot (n+1)^n$.  So we have
        \[ (n+1) \cdot n! \leq (n+1) \cdot (n+1)^n \]
        and
        \[ n! \leq (n+1)^n \]

        Teacher's proof
        \begin{align}
          (k+1)! &= (k+1) \cdot k! \\
                  \leq (k+1) k^k \\
                  \leq (k+1)^{k+1}
        \end{align}
        Hence by induction principle, $n! \leq n^n$ for all $n \in \mathbb{N}  $
      }
    }
    \ex{$2^n \leq n!$} {
      Using Induction, prove that $2^n \leq n!$ for $n \geq 4$

      \myproof {
        \underline{Base Case: } $2^4 = 16$ and $4! = 24$.  So $16 \leq 24$
        \underline{Induction Step } Assume that $2^k \leq k!$ for seom
        $k \geq 4$.  Then $s^{k+1} = 2^k \cdot 2 \leq k! \cdot 2 \ leq
        k! \cdot (k+1) = (k+1)!$
      }
    }
    \thm{ Strong (Complete Induction } {
      Let $P(n)$ be a statement that depends only on $n \in N$
      Suppose the following hods
      \begin{enumerate}
        \item $P(1)$ is true
        \item If $P(m)$ is true for all $m \leq k$, then $P(k+1)$
          is true.
      \end{enumerate}
      Then $P(n)$ is true for all $n \in \mathbb{N}$
    }
    \myproof {
      $Q(n) = P(1) \wedge P(2) \wedge P(3)  \dots \wedge P(n) = 
      \wedge^n_{i=1} P(i)$

      \underline{Base Case: } $Q(1) = P(1)$, which is true

      \underline{Induction Step } Suppose $Q(k)$ is true for some
      $k \in \mathbb{N}$, i.e. $P(1) \wedge  \dots P(k)$ is true.  Which
      is the same as saying that $P(m)$ is true for all $m \leq k$.
      By assumption, $P(k+1)$ is true.  Hence, $Q(k) \wedge Q(k+1)$
      is also true.  So $Q(k+1)$ is true.  Thereoffre by Induction
      Principle, we are done
    }
    \ex{HOMEWORK} {
      Let $\{a_n\}_{n=0}^{\infty}$ be a sequence such that $a_0=1,
      a_1=1, \dots a_n = a{n-1} + a_{n-2}$ for $n \geq 2$.  Prove that
      $a_n \leq 2^n$ for all n.

      \emph{HINT:  Use complete induction}

      \myproof {
        \underline{Base Case: } Start with $n=2$.  Then $a_2 = a_0 + a_1 = 1+1 = 2 \leq 2^2$

    \underline{Induction Step: } Suppose $a \leq 2^m$ for all $m \leq k$.  Then, $a_{m+1} = a_m + a_{m-1} \leq 2^m + 2^{m-1} \leq 2^m +2^m = 2^{m+1}$
      }

    }
%    \nt{ If there are $a_n$ and $a_{n-1}$ kinda thing, use strong
%      induction
%    }

    \ex{HOMEWORK} {
      Show that $\sum^{n}_{i=1} i^2 = \frac{n(n+1)(2n+1}{6} $
    }
    \ex{HOMEWORK} {
      \emph{HINT:  Use complete induction}

      Let $a_1, a_2$ be a sequence with $a_1=1, a_2=5$ with
      $a_n = a_{n-1}+2a_{n-2}$ for all $n \geq2$.  Show that
      $a_n = 2^n+(-1)^n$ for all $n \geq 1$
    }
\section*{20230627 - Induction and Functions}%
  \subsection*{Complete Induction - Recap}%
  \begin{itemize}
    \item Base Case: Either 1 or $n_0$ is true
    \item Induction Step: If $P(m)$ is true $\forall m \leq k \implies
      P(k+1)$ is true, then $P(n)$ is true for all $n \in \mathbb{N}$
      \ex{HOMEWORK from yesterday} {
        \myproof {~\\
          \underline{Base Case: } show n=1, n=2

          \underline{Induction Step: } See Picture
        }
        
      }
  \end{itemize}

  \subsection*{Functions}%
    \begin{itemize} Plan
      \item  Notations
      \item definitions
    \end{itemize}

    If A and B are sets, then we can define funtions from A to B

    Assume both A and B are the $\mathbb{R}$
    \dfn{ Function --- Informal }{
      Rule that assigns (certain) elements in A to elements in B
    }

    \dfn{ Domain }{
      All the numbers in $R$ that $f$ assigns a number to it

      Notation --- Dom($f$ )

      $f(x)$ for $x \in $ Dom($f$ ) is the number that $f$ assigns to $x$ 

      \underline{A Function must always contain a Domain}
    }

    \dfn{ Codomain }{
      Possible range of values for B, e.g. $\mathbb{R}$
    }
    \dfn{ Range }{
      All values in $g \in \mathbb{R}$ such that there is an $x \in $
      Dom($f$ ) such that $f(x)=y$
    }
    \ex{sin} {
      \begin{itemize}
        \item Domain:  $\mathbb{R}$
        \item Codomain:  $\mathbb{R}$
        \item Range: $[-1,1]$
      \end{itemize}
    }
    \ex{$f(x) = \frac{1}{x}$ } {
      \begin{itemize}
        \item Domain: x, $x \in \mathbb{R}, x \neq 0$
      \end{itemize}
    }
    \subsection{Special Functions}%
    \begin{itemize}
      \item Polynomials: A functions from $\mathbb{R} \Rightarrow \mathbb{R}$
        is called a polynomials function if 
        \[P(x) = a_nx^n + a_{n-1}x^{n-1} \dots, \forall x \in \mathbb{R}\]
        for some $a_n, a_{n-1}, \dot, a_0 \in \mathbb{R}$

        Degree of $P$ is $n$, i.e. the highest power of $x$ such that its
        coeffficient is non-zero

        E.g., $t^2 + t + \pi/2$: degree $p=2$
      \item Rational 
        \dfn{ Rational Function}{
          A function  $f$ is valled a rational function if
          \[ f(x) = \frac{p(x)}{q(x)}  \]
          where $p,q$ are polynomial function, and $
          x \in \{x:x \in \mathbb{R}, q(x) \neq 0  \}$
        }
        E.g. $f(t) = \frac{t^2+1}{t^2-1} $.  Domain = $\{t: t \in
        \mathbb{R}, t \neq \pm 1\}$
        
    \end{itemize}
    \subsection{Operations on Functions}%
    \begin{itemize}
      \item Multiply by Constant:  Let $f$ be a function, then we define
        the function $cf$ ffor some $ c\in \mathbb{R}$ as
        \[ (cf)(x) = c \cdot(f(x)) \]

        Ex: $f(x) = x \forall x \in \mathbb{R}$, then
        $(2f)(x) = 2x, \forall x \in \mathbb{R}$

        \underline{Dom(cf) = Dom(f)}
      \item Addition: If $f$ and $g$ are ffunction, then we define
        $(f + g)(x)$ as $f(x) + g(x), \forall x \in \text{ Dom }(f)
        \cap \text{ Dom }(g)$

        Ex: $f = 1/x, g = 1/(x-1), (f+g)(x) = 1/x + 1/(x-1)$

      \item Multiply: If $f$ and $g$ are functions we define
          \[ (f \cdot g)(x) = f(x) \cdot g(x) \]

          Domain ($f \cdot g) = \text{ Dom }(f) \cap \text{ Dom }(g)$

      \item Divide: If $f$ and $g$ are functions we define
          \[ (f / g)(x) = f(x) / g(x) \]

          Domain ($f / g) = \text{ Dom }(f) \cap \text{ Dom }(g)
          \cap \{x: g(x) \neq 0 \}$

      \item Composition: If $f$ and $g$ are functions we define
          \[ (f \circ g)(x) = f(g(x)) \]
          Domain ($f \circ g) = \text{ Dom }(g) \cap \{x: g(x) \in
          \text{ Dom }(f) \}$
    \end{itemize}

    \dfn{ Function }{
      A function $f$ is a subset of $\mathbb{R} \times \mathbb{R}$
      such that if $(a,b) \in f$ and $(a,c) \in f$ then $b=c$

      \underline{Domain}: Dom(f) = $\{ a: (a,b) \in f$, for some
        $b \in \mathbb{R}$

      \nt{
        From the definition, it is clear that there is a unique
        $y \in \mathbb{R}$ subthat $(x,y) \in f$ for $x \in \text{ Dom }
        (f)$.  We can define this unique $y$ as $f(x)$
      }
      \nt{ It is useful to thinkoff functions as rules}
    }

    \ex{} {
      Let $f$ amd $g$ be functiond with domain $\mathbb{R}$.  Let
      this set be F
      \begin{enumerate}
        \item Show that $f+g \in F$ and $f \cdot g \in F$
        \item Define $\theta(x) = 0, \forall x \in \mathbb{R}$.  
          and  $\theta(x) = 1, \forall x \in \mathbb{R}$.  


          Clearly $\theta$ and $\iota \in F$.  Show that $f + \theta = $
          and $f \cdot \iota = \iota \cdot f \ f$

        \item $g = (F, +, \cdot, \theta, \iota)$.  What properties P1-P9
          does g satisfy
        \item $P = \{f: f \in F, f(x)>0 \forall  x \in \mathbb{R}\}$.
          Does g (from directly above) satisfy P10-P12
        \item How do we define -f?
      \end{enumerate}
    }

\section*{20230628 - Graphs}
  \subsection{Intervals}%
  \begin{itemize}
    \item Algebraic notion of real number ??
    \item Geomtric notion: Real Number line
    \item Intervals: $[a,b], (a,b), [a,b), (a,b]$
    \item Because $\mathbb{R}$ is infinite: $([a,\infty), (- \infty,a),
      (-\infty, a), (- \infty, a]$
    \item Absolute Value: |x-y| is the distance between x and y on the
      real line
    Assume $\epsilon > 0 \text{ and } \epsilon < a$, then
    \begin{align}
      (a - \epsilon, a + \epsilon) &= \{ x: a - \epsilon < x < a + \epsilon 
      &= \{ x: - \epsilon < x - a < \epsilon 
      &= \{ x: |\epsilon -a | < \epsilon 
    \end{align}
      I.e., All points at a distance less than $\epsilon$ frfom $a$ on
      $\mathbb{R}$
  \end{itemize}

  \subsection{The Plane}%
    \begin{itemize}
      \item $\mathbb{R}^2 = \mathbb{R} \times \mathbb{R} = 
        \{(x,y): x \in \mathbb{R}, y \in \mathbb{R}$
        \item dist((a,b), (c,d)) = $\sqrt{(a-c)^2 + (b-d)^2}$
    \end{itemize}

  \subsection{Graph of Functions}%
    \begin{itemize}
      \item A grapgh is a geometric representation of a function
      \item The graph of a function $f$ is
        s\[ \{(x, f(x)): x \in \mathbb{R}\} \subseteq \mathbb{R}^2 \]
    \end{itemize}
    \ex{$(x) = x$} {
       graph of a line
    }
  \item Graph(f) := $\{ (x,y): x \in \mathbb{R}, y = f(x) \}$

    \subsection{Misc}%
    \begin{itemize}
      \item $\frac{p}{q} $ is lowest terms is the same as gcd(p,q) = 1.
        We say p and q are relatively prime
    \end{itemize}
  
    
\end{document}
